Dear Reader, though you encounter this section first, I write it as the final
piece of my dissertation. In some sense I have saved the best for last, that is
to say, the names of all the people who have made this body of work possible. In
some particular order, I would like to thank my parents, Ed and Pat, without
whom I would not exist and whose infinite patience, care, and concern have
supported me for all these years. I also wish to thank my sister, Rachel, who
had markedly less patience with me, but was otherwise on par. Next, a great deal
of credit is due to my girlfriend, Margaret Shumbarger, who did not have the
convenience of living 700 miles from my idiosyncrasies, but nevertheless did
everything to help me through this process (it was not easy, for either of us).

My advisor, John Foster, has been both a great mentor and a great friend for all
of our time together at the University. While my knowledge has greatly
grown for having known him, his sincere kindness and earnest nature are just as
important. To my current and former labmates: Brandon Weatherford, Brad
Sommers, Aimee Hubble, Eric Gillman, Sarah Gucker, Kapil Sawlani, Alex Englsebe,
and Neil Arthur; your camaraderie was invaluable, and I shall always be thankful
for that. Though I never met them in person, I should also acknowledge the
financial outlays of the Department of Energy (Fusion Energy Science Contract
DE-SC0001939) and NASA (Training Grant NNX09AK95H) which made all of the
presented work possible.

Finally, I should express my gratitude for the student radio station, WCBN, and
all of its members. As the physical manifestation of mayhem, exuberance, and
contemplation (simultaneously), it has been the bedrock of my imagination. You
may be tempted to think that there is little which crosses over between nuclear
engineering and radio, but you would be wrong. That said, I think that science
(and the whole world) could benefit from a trip down into the deepest recesses
of Sun Ra's synthesizer. It'd clean out the waxy buildup.
