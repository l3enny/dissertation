The measurements of the metastable dynamics may be used to better understand the
development of the \acs{rpnd} by the use of a plasma model. Working from the
metastable densities, it may be possible to predict electron characteristics,
such as temperature, density, as well as the densities of other excited states.
The ideal model would solve the Boltzmann equation (equation~\ref{eq:boltzmann})
for each species, over the entire geometry, in all dimensions, for as long as
was required to reach equilibrium.

Unfortunately, these requirements are somewhat problematic. Scale lengths of 10
$\mu$m are required in order to resolve the sheath effects, resulting in
approximately $2\times10%{13}$ spatial cells. Conservatively, the time steps
would be about 1 ns in length. At a minimum, the system required about five
minutes to reach equilibrium, thus necessitating $3\times10^{11}$ solution
steps. The largest velocity can be estimated as an electron accelerated across
the full applied potential (6.6 keV), and the lowest velocity would be room
temperature (0.04 eV). This produces a velocity discretization of about
$6\times10^7$ cells. Thus the size of the parameter space in question is about
$3.6\times10^{32}$. Given that the fastest computer in the world operates at 39
petaflops, a calculation of this magnitude would take around 0.3 billion years.
Of course, this is for only a single particle species, the total number in the
system is runs in the dozens (not including the impurities).

\section{Model Development}

It should be apparent that the Boltzmann equation must be simplified in order to
model the system in a reasonable amount of time. As discussed in
chapter~\ref{chp:theory}, the most common approach to this is the use of moments
of the Boltzmann equation which drastically simplifies the velocity terms.
Frequently, the moments of the Boltzmann equation are used to develop various
fluid approximations for plasmas \cite{Chen1984} (e.g.\ the two fluid equations,
the magnetohydrodynamic equations, etc.). This approach has been tremendously
successful in the description of everything from plasma display panels
\cite{Rauf1999b} to interstellar plasmas \cite{Linde1998}.

There are some limitation to the capabilities of these fluid descriptions. For
one, they require some assumption on the form of the \acs{eedf}. Often, the
distribution is assumed to be a Maxwell-Boltzmann or Druyvesteyn, depending on
the plasma conditions. In others, an approximate solution of the Boltzmann
equation may be used to tabulate rate coefficients as a function of the mean
energy \cite{Hagelaar2005}. In addition to this issue of the \acs{eedf}, fluid
models in large or complex geometries can still be quite computationally
expensive. This can limit the number of species and reactions which can be
addressed \cite{Lieberman2005}.

In order to obtain an estimate of the detailed dynamics which occur as the
\acs{rpnd} develops, it is necessary to consider additional simplifications of
the Boltzmann equation. One such possibility is the use of a global model where
the spatial dependence of the plasma parameters is assumed. This allows global
model simulations to ignore the geometry of the system and focus on the particle
interactions for long periods of time with reasonable computational
requirements.



The metastable measurements from the previous chapter showed little axial
variation for the majority of the operating conditions. Radial variations in the
emission intensity and excited state density have been observed
\cite{Vasilyak1994, Weatherford}, however the mechanism for this is poorly
understood.



\section{Distribution Effects}

\section{Pulse Shape Changes}

\section{Plasma Parameters}

\section{Summary}

