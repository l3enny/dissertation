Though the line-integrated metastable densities are one of only a few
measurements made in the development of the \acs{rpnd}, they only provide a
limited view of what is happening. In addition to the metastables are ions,
electrons, a vast array of other excited states, and the electric fields. In an
effort to expand on the details of what is occurring within the \acs{rpnd}, it
is desirable to develop a model which can infer other properties from the
metastable measurements. This is possible, because the electrons which gain
energy from the electric fields are those which excite the metastables, other
excited states, and ions. The ideal model would solve the Boltzmann equation
(equation~\ref{eq:boltzmann}) for each species, over the entire geometry, in all
dimensions, for as long as was required to reach equilibrium.

Unfortunately, these requirements are somewhat problematic. Scale lengths of 10
$\mu$m are required in order to resolve the sheath effects, resulting in
approximately $2\times10^{13}$ spatial cells. Conservatively, the time steps
would be about 1 ns in length. At a minimum, the system required about five
minutes to reach equilibrium, thus necessitating $3\times10^{11}$ solution
steps. The largest velocity can be estimated as an electron accelerated across
the full applied potential (6.6 keV), and the lowest velocity would be room
temperature (0.04 eV). This produces a velocity discretization of about
$6\times10^7$ cells. Thus the size of the parameter space in question is about
$3.6\times10^{32}$. Given that the fastest computer in the world operates at 39
petaflops, a calculation of this magnitude would take around 0.3 billion years.
Of course, this is for only a single particle species, the total number in the
system is runs in the dozens (not including the impurities).

\section{Model Development}

It should be apparent that the Boltzmann equation must be simplified in order to
model the system in a reasonable amount of time. As discussed in
chapter~\ref{chp:theory}, the most common approach to this is the use of moments
of the Boltzmann equation which drastically simplifies the velocity terms.
Frequently, the moments of the Boltzmann equation are used to develop various
fluid approximations for plasmas \cite{Chen1984} (e.g.\ the two fluid equations,
the magnetohydrodynamic equations, etc.). This approach has been tremendously
successful in the description of everything from plasma display panels
\cite{Rauf1999b} to interstellar plasmas \cite{Linde1998}.

There are some limitation to the capabilities of these fluid descriptions. For
one, they require some assumption on the form of the \acs{eedf}. Often, the
distribution is assumed to be a Maxwell-Boltzmann or Druyvesteyn, depending on
the plasma conditions. In others, an approximate solution of the Boltzmann
equation may be used to tabulate rate coefficients as a function of the mean
energy \cite{Hagelaar2005}. In addition to this issue of the \acs{eedf}, fluid
models in large or complex geometries can still be quite computationally
expensive. This can limit the number of species and reactions which can be
addressed \cite{Lieberman2005}.

In order to obtain an estimate of the detailed dynamics which occur as the
\acs{rpnd} develops, it is necessary to consider additional simplifications of
the Boltzmann equation. One such possibility is the use of a global model where
the spatial dependence of the plasma parameters is assumed. This allows global
model simulations to ignore the geometry of the system and focus on the particle
interactions for long periods of time with reasonable computational
requirements.

Here, a global model will be developed to describe the evolution of the plasma
parameters along the measurement chords during the first pulse, and before the
reflection. As was seen in the metastable measurements, there was little axial
variation over the length of the discharge. Thus, as far as the model is
concerned, variations across the beam radius are negligible. However, it has
been noted that a similar \acs{fiw} \cite{Vasilyak1994} and the same \acs{rpnd}
\cite{Weatherford2012} exhibit radial variations in emission intensity, electron
density, and metastable density. Furthermore, these variations appear to depend
strongly on the operating pressure. Unfortunately, the cause of this is not
clearly understood, though it has been suggested that high-energy electrons from
the walls may be responsible \cite{Weatherford2012a}. Lacking any empirical,
theoretical, or numerical results that provide the evolution of the radial
profile during the discharge, it is necessary to assume one. In this case, the
plasma will be assumed to be uniform across the diameter of the discharge. This
assumption will affect the accuracy of the inferred plasma parameters, however
future improvements in the understanding of these radial variations can be
easily incorporated into the described model.

\subsection{Continuity Equation}

The global model begins with equation~\ref{eq:cont}, the continuity equation.
Having assumed that the spatial variations are zero, the equation reduces to,
\begin{equation}
  \frac{d n_\alpha}{dt} = G_\alpha - L_\alpha,
  \label{eq:zdmcont}
\end{equation}
where $\alpha$ identifies the particle species, $G$ is the gain term, and $L$ is
the loss term. The gain and loss terms represent a large range of possible
processes which are determined, in part, by the species under consideration. In
this model, only helium, excited helium states, helium ions, and electrons will
be treated. While impurities and dimers are very much a part of the discharge,
this occurs on time scales that are relatively long in comparison to the
excitation period. As was noted in chapter~\ref{chp:metastables}, only 140 ns
elapse before the reflection arrives at the plasma, while the e-folding time of
the fastest decay is about 25 $\mu$s.

Given the species of the system, there are several processes that should be
given consideration for inclusion in the model.
\begin{enumerate}
  \item Electron impact ionization 
  \item Electron impact (de)excitation 
  \item Atomic impact (de)excitation
  \item Atomic excitation transfer
  \item Dielectronic recombination
  \item Three-body recombination
  \item Radiative decay
  \item Diffusion
\end{enumerate}
As with the impurities and dimer formation, diffusion occurs on a much longer
time scale, and thus is neglected. Three-body recombination in the volume of the
discharge is not important at the estimated temperatures and densities
\cite{Lieberman2005}, therefore this too is neglected. Dielectronic
recombination is also quite rare, however the process was incorporated as part
of the early models and thus maintained through the final revision. Inter-atomic
excitation and de-excitation is not generally considered important given the low
energies of the atoms in a discharge. However, excitation \emph{transfer} can be
an important process in gaseous discharges \cite{Lieberman2005}, making their
inclusion necessary. Also important are the electron impact ionizations and
excitations, as well as the radiative decay of the atoms.

Given these processes, it is possible to rewrite equation~\ref{eq:zdmcont} as,
\begin{multline}
  \frac{dN_i}{dt} =   n_e \left[       \sum_{j\neq i} N_j K^e_{ji}(T_e) 
                                 - N_i \sum_{j\neq i}     K^e_{ij}(T_e) \right]
                        + \left[       \sum_{j\neq i} N_j K^o_{ji} 
                                 - N_i \sum_{j\neq i}     K^o_{ij}      \right] \\
                    + N_g \left[       \sum_{j\neq i} N_j K^a_{ji} 
                                 - N_i \sum_{j\neq i}     K^a_{ij}      \right].
  \label{eq:gcont}
\end{multline}
In these equations, the subscripts of $i$ and $j$ represent states of helium,
$N$ is a state density, $K$ is a rate coefficient, $T_e$ is the electron
temperature, and $N_g$ is the neutral helium density. The first subscript of the
rate coefficients represents is the initial excited state while the second
coefficent represents the final excited state. Therefore, $K_{ij}$ represents a
process that depopulates state $i$ in favor of state $j$.

This equation is split into three sets of processes, represented by the
superscripts of the rate coefficients: $e$ - electronic, $o$ - optical, and $a$
- atomic. The first bracketed term on the right hand side contains all the rate
coefficients for electron impact excitation and de-excitation, including
ionization processes. The second bracketed term contains the rate coefficients
for optical transitions in and out of the state. The final bracketed term
contains the gains and losses as a result of excitation transfer caused by
collisions with the ground state. Collisions between excited states are
neglected given their relatively low densities.

The rate coefficients in equation~\ref{eq:gcont} are compiled from a number of
different sources. This is particularly straight forward in the case of the
optical and atomic transitions, as neither features any dependence on the
\acs{eedf}. The optical transition rates and level energies were obtained from
the NIST Atomic Spectra Database \cite{Kramida2012}. The excitation transfer
rate coefficients were from the studies of Catherinot and Dubreuil
\cite{Catherinot1981, Dubreuil1980}. These coefficients only covered the
transitions of $\Delta n=0$ for $n=3,4$ and no constants were found for other
$n$ or $\Delta n\neq 0$.

The semi-empirical relations derived by Ralchenko et al. \cite{Ralchenko2008}
were used to calculate the electron (de)excitation and ionization cross sections
through $n=4$. These represent the most accurate set of cross sections available
for neutral helium and have a quoted accuracy of 10-30\% for $\Delta S=0$, and
$\ge30$\% for $\Delta S \neq 0$. These cross sections can be used to calculate
the rate coefficients for each reaction using equation~\ref{eq:rate}. However,
this poses the problem of what kind of energy distribution to use.

\subsection{Distribution Effects}

Starikovskaia and Starikovskii \cite{Starikovskaia2001} showed that the
\acs{eedf} changes rapidly over the duration of a \acs{fiw} in nitrogen. It is,
therefore, reasonable to expect that the same would apply to electrons in a
helium \acs{rpnd}. Furthermore, their results showed that a large number of
runaway electrons were present ahead of the discharge front. They found that the
two-term Boltzmann solvers commonly used in fluid codes did not adequately
capture the dynamics of the nitrogen \acs{fiw} front, but they did fare well in
the plasma after the front had passed.

As has been discussed, there are several differences between \acs{fiw}s and
\acs{rpnd}s. Likewise, rare gas discharges are also quite different from those
in molecular gases. It was therefore desirable to conduct a similar study of the
\acs{eedf} in the helium \acs{rpnd} under consideration. Three different
approaches were compared: an assumed Maxwell-Boltzmann distribution, a two-term
Boltzmann solver, and a particle-in-cell (\acs{pic}) code with Monte-Carlo
collisions.

BOLSIG+ was used as the Boltzmann solver \cite{Hagelaar2005}. The input cross
sections came from the work of Phelps \cite{Phelps2002}. The solutions of the
Boltzmann equation were calculated for a range of reduced electric fields from
1-500 Td. The mean energy calculated by BOLSIG+ was used to generate the
Maxwell-Boltzmann distributions for comparison.

\acs{pic} codes approach the simulation of plasmas from a different direction
than a direct solution of the Boltzmann equation. Instead, they directly

\subsection{Energy Equation}



\section{Pulse Shape}



\section{Plasma Parameters}



\section{Summary}



