The repetitively-pulsed nanosecond discharge \acs{rpnd} is a low temperature
plasma with several properties which make it amenable to novel applications.
Some of these properties include its production of a large uniform discharge (on
the order of liters), negligible gas heating, and operation over a wide range of
pressures \cite{Starikovskaia2001}. Though the \acs{rpnd} has only recently been
enabled by new semiconductor technology, it shares a long history with other
pulsed discharges. This includes the fast ionization wave (\acs{fiw}),
streamers, sparks, and lightning \cite{Loeb1965}.

Despite this long history, in-depth study of these phenomena has been
challenging. A review of the literature found that the important dynamics of the
\acs{rpnd}, which can occur in a matter of nanoseconds, were not well-explored.
This can be attributed to the time scales involved which present a number of
technical difficulties. Separately, the high electric fields and the substantial
collisionality within the \acs{rpnd} precludes the use of many traditional
plasma diagnostic techniques. Thus, most studies have focused on measurements of
the plasma properties after the pulse or averaged over the pulse.

However, as with most plasmas, the properties of the \acs{rpnd} are determined
by the coupling between the applied electric field and the gas. As this coupling
is a result of fundamental processes which occur during the nanosecond pulse,
such as electron acceleration and collisional excitation, the \acs{rpnd} can not
be understood without more detailed consideration of the dynamics which occur
during its initial phases.

\section{Overview of Results}

To that end, the necessary theory was developed to describe the \acs{rpnd}. This
began with the Boltzmann equation and Maxwell's equations, the basic equations
which can be used to describe any ionized gas. Several reductions of the
Boltzmann equation were presented in the form of moments. These moments averaged
over the velocity space of the probability distribution in order to simplify the
solution of the equations.

Following this, the criteria for a plasma were presented. A plasma is distinct
from an ionized gas in the sense that its behavior is dominated by its
electromagnetic properties. Thereafter, the initiation of a plasma from one or a
few electrons was explained for the two primary cases. The first, the Townsend
mechanism, occurs for relatively small electric fields, long periods of time,
and does not involve the formation of appreciable space charge. In contrast, the
streamer mechanism, involves a high electric field, develops over a shorter
period of time, and is influenced by the formation of large regions of space
charge.

It is the properties of the streamer mechanism which bear the most resemblance
to the \acs{rpnd}. However, the streamer model does not automatically account
for the uniform nature of the \acs{rpnd}. Instead, this was described by the
work of Levatter and Line \cite{Levatter1980}. They demonstrated that a streamer
can develop in a uniform manner provided a sufficient density of pre-pulse
electrons. Calculations for the subsequent experimental conditions found that
the natural electron density was sufficient to guarantee uniform breakdown at
all operating pressures. The theoretical discussion concluded with an
explanation of atomic structure, spectroscopic notation, and spectral
lineshapes.

Subsequently, an experimental \acs{rpnd} was described. It was generated by
repetitive positive voltage pulses in ultra-high purity helium at several
different pressures. The discharge geometry was of a coaxial design similar to
the \acs{fiw} studies reviewed by Vasilyak et al. \cite{Vasilyak1994}. Visual
observations and the ease with which breakdown was obtained suggested that the
plasma was most stable at a pressure of 4.0 Torr. However, estimates of the
energy coupling from the current-voltage measurements indicated that greatest
energy coupling occurred at 1.0 Torr, with a peak value of 5.5 mJ. This was
consistent with previous experimental measurements of the energy coupling in
\acs{rpnd}s and \acs{fiw}s.

Laser-absorption spectroscopy (\acs{las}) of the 2$^3$S, or triplet metastable,
state of helium was the primary means by which the nanosecond discharge dynamics
were measured. This technique produced measurements of the temperature and
line-integrated density of the triplet metastable state over the duration of the
pulse. The triplet metastable level because it can act as a large energy
reservoir in helium discharges and can increase the charged particle population
via associative ionization and Penning ionization. The use of an active optical
diagnostic, meant that the time resolution of the measurements were only limited
by the bandwidth of the detector, 5 ns.

The results of the \acs{las} confirmed that no gas heating was occurring in the
\acs{rpnd}. The accuracy of the temperature measurements varied with respect to
the metastable density, but was generally about $\pm50$ K. The largest detected
line-integrated densities, prior to the reflected pulse, occurred for the 4.0
Torr condition at a value of about 5.9$\times10^{16}$ m$^{-2}$. Assuming a
uniform density distribution across the discharge tube, this is equivalent to a
density of 1.8$\times10^{17}$ m$^{-3}$. A significant number of metastable atoms
persisted between pulses for pressures of 1.0 Torr and lower.

There were no observable trends in the metastable densities with respect to
axial location for pressures of 4.0 Torr and lower. This provides additional
confirmation of the homogeneous and volume-filling nature of the \acs{rpnd}. In
contrast, results at 8.0 and 16.0 Torr showed a clear decline in metastables
densities as a function of distance from the anode. This is believed to be akin
to the wave attenuation in \acs{fiw}s, described by Vasilyak et al.
\cite{Vasilyak1994}.

Long duration measurements of the metastable densities revealed the primary
decay processes in this discharge geometry. Metastable destruction at low
pressures was dominated by associative ionization; indicated by deviations from
a purely exponential decay. As the pressure of the system was increased, the
three-body reaction (leading to the formation of helium dimers) became the
dominant destruction process. Decay constants significantly larger than
those reported by Deloche et al. \cite{Deloche1976} and Phelps and Molnar
\cite{Phelps1953} suggest that gaseous impurities played a significant role in
metastable destruction.

A global model was then developed in order to infer various other plasma
parameters from the metastable measurements. The development of the model
included an analysis of the likely electron energy distribution functions
(\acs{eedf}s) in the \acs{rpnd} with a series of zero-dimensional
particle-in-cell (\acs{pic}) simulations. For fields of 100 Td and less, the
\acs{pic} simulations showed good agreement with Maxwell-Boltzmann distributions
of the same mean electron energy. However, above 300 Td, the Maxwell-Boltzmann
distributions exhibited significantly fewer high-energy (greater than 100 eV)
electrons. Better agreement with the \acs{pic} simulations at these field
strengths were obtained with solutions of the two-term expansion of the
Boltzmann equation. However, as a result of significant disagreements at lower
electric fields, the global model assumed a Maxwell-Boltzmann distribution for
the entirety of the simulations.

The global model simulations which best matched the measured metastable density
trends required the use of an exceptionally long electric field, 40 ns in
length, compared to the 25 ns length of the applied pulse. Upon inspection of
the optical emissions from the plasma, a return stroke was identified that
appeared to explain this longer-than-expected excitation. Based on previous
results in \acs{fiw} research \cite{Starikovskaia1998}, an alternative
explanation was also proposed in which a beam-like electron population formed in
the \acs{rpnd} and its relaxation was responsible for the extended excitation.

Using the long excitation period with the global model produced an excellent
match to the measured metastable densities, both during the pulse and afterward.
Particularly good agreement was obtained at 4.0 and 8.0 Torr conditions. Small
deviations were observed at 1.0 Torr--the metastable density appeared to rise
too quickly during the pulse, and too slowly after the pulse. This provided a
second hint that energetic electrons may be present in the system, specifically
at the lower pressure conditions.

This was reinforced by the magnitude of the electric fields and inferred
electron temperatures necessary to match the metastable densities. At 1.0 Torr,
the peak electric field was 346 Td and the peak electron temperature was
estimated at 74.6 eV. This suggests that the global model is likely inaccurate
for this condition given the disagreement between the Maxwell-Boltzmann
distribution and the \acs{pic} simulations at field values over 300 Td.
Additionally, the electron temperature is far in excess of what can be
considered reasonable for an equilibrium population of electrons in a low
temperature plasma.

In contrast, the global model predictions at 4.0 and 8.0 Torr were better
aligned with previously reported values. The electron temperatures are similar
to those recently obtained for a \acs{fiw} \cite{Takashima2011}, as well as
those predicted from rate coefficient calculations \cite{Aleksandrov2007}.
Likewise, the electric fields are more reasonable than those predicted at 1.0
Torr, though they fall in the intermediate region between 100 and 300 Td.

Interesting to note was a delay between the peak electric field and the highest
ionization rate in the plasma. This was interpreted as the time required for the
seed electron population to reach ionization-relevant temperatures. In a
simplified system, the ionization rate is an exponential function of time,
thus the density grows quickly thereafter.

The global model was designed to consider a total of 32 different species and
535 different reaction processes, including optical transitions. This provided
the opportunity to compare simulated plasma emissions generated by the global
model with those measured from the experimental \acs{rpnd}. Emission
measurements of the \acs{rpnd} covered ten detectable transitions in the
visible spectrum. These were used to estimate the wave velocities of the
\acs{rpnd}: 1.7-3.0$\times10^7$ m/s at 8.0 Torr, and 0.7-1.5$\times10^7$ at 16.0
Torr. All other conditions had wave velocities that were greater than
5$\times10^7$ m/s.

Afterward, an initial attempt was made to determine the evolution of the
electron temperature from a series of Boltzmann plots. The Boltzmann plots from
the simulated emissions and measured emissions both resulted in temperature
estimates of about 0.5-0.6 eV. The temperatures estimated from the simulated
emissions showed significant disagreement with the actual simulated
temperatures. This indicated that the use of Boltzmann plots for the \acs{rpnd}
is almost certainly flawed.

The reason for this lies in the assumption of partial local thermodynamic
equilibrium in the plasma. This requires that the excited state populations used
in the Boltzmann plot are in equilibrium with the electrons \cite{Kunze2009}.
However, the \acs{rpnd} likely develops too rapidly for this equilibrium to take
place. This conclusion was reinforced by an examination of the individual
Boltzmann plots for a series of times after the pulse.

Subsequently, an attempt was made to determine the electron temperature via a
coronal model and the use of line ratios \cite{Griem2005}. Three line ratios
were considered however one was immediately rejected for lack of detectable
emissions. The second, which compared the 4$^3$S-2$^3$P$\odd$ transition to the
4$^1$D-2$^1$P$\odd$ transition was used with little success. Estimates based on
the measured emissions produced unrealistic temperatures, and the estimates from
the simulated emissions did not agree with the actual temperatures from the
global model.

The second line ratio which compared the 3$^3$S-2$^3$P$\odd$ transition to the
3$^1$S-2$^1$P$\odd$ transition produced more promising results. The estimates
from the simulated emissions matched the global model temperatures, which
demonstrated that the line could potentially work provided the global model
results are applicable.

However, the estimates from the measured emissions were less successful. For
most points in time, the two transitions were too dim to obtain a reliable
ratio. In the cases that a temperature could be estimated, the results indicated
a peak electron temperature of about 5 eV, followed by a quick cooling to 0.5
eV. As the coronal model relies on the assumption of a Maxwell-Boltzmann
distribution, this approach will almost certainly produce misleading and
incorrect results at pressures of 1.0 Torr and below.

Additional comparisons were made between the simulated and measured emissions
for the 3$^1$D-2$^1$P$\odd$ transition, at 1.0, 4.0, and 8.0 Torr. Some general
features, such as the timing of the peak intensity, and the relative intensities
between pressures were consistent between the two. However the simulated
emissions generally decayed away faster after the peak intensity as compared to
the measured emissions. This suggested a prolonged excitation period in the
\acs{rpnd}, again hinting at the presence energetic electrons.

Furthermore, the decay rate of the measured emissions had a noticeable pressure
dependence. This suggested an atomic process was responsible, one which was not
accounted for in the global model. The metastable measurements suggested that
Penning ionization of impurities was an important loss mechanism. As a result,
it is believed likely that the 3$^1$D state was also affected by collisions with
gaseous impurities. This would explain the pressure dependence observed in the
measured emissions.

Finally, the effects of radiation trapping in the \acs{rpnd} were investigated
as another possible source of extended excitation. Observable effects of
radiation trapping were confirmed by a comparison of the simulated and measured
emissions from the 3$^1$P$\odd$-2$^1$S transition. Subsequent calculations
showed that the effective lifetime of the 3$^1$P$\odd$-1$^1$S transition could
be extended by a factor of nearly $10^5$ for the geometry in question. The
increased residence time of this energy increases the chance that it may be
transferred to another excited state or lead to ionization via collisions.
This suggests that radiation trapping in the \acs{rpnd} may lead to an increase
in the excited state and charged particle densities, similar to the metastable
states.

\section{Future Work}

There are several opportunities to improve on the work presented here. From an
experimental perspective, the metastable density measurements could be improved
by the use of a more sensitive detector. This would decrease the minimum
detectable metastable density and potentially provide pre-pulse metastable
densities for the other operating pressures. It would also be desirable to
revisit the existing metastable density measurements along various chords of the
discharge cylinder. This would provide radial density profiles via inversion
techniques.

The emission measurements could also stand to be improved by the use of a more
efficient grating, a more efficient photocathode, or both. These changes should
be made with measurement of the 706 and 728 nm lines in mind, as these appear to
be the most promising for transitions for use in optical electron temperature
measurements. Further improvements to the optical temperature measurements could
be made with the development of an appropriate collisional radiative model and
perhaps the use of alternative \acs{eedf}s in the calculation of the rate
coefficients.

The experiment could also be improved with the relatively simple addition of a
capacitive probe for sensing of the electric field within the plasma, similar to
that used by Takashima et al. \cite{Takashima2011}. The electric field
measurements would allow for confirmation of the return stroke which was
observed in the emissions data. The probe may also be used to determine if a
persistent electric field exists after the pulse.

The electric field information could then be incorporated into the global model
for more accurate predictions of the \acs{rpnd} properties. Other improvements
to the global model would include the addition of gas impurities and reactions
in response to the evidence that they contribute to the \acs{rpnd} dynamics.
Furthermore, the global model should be modified to include radiation trapping
effects for resonant transitions.

However, the global model is still fundamentally limited in how it handles the
\acs{eedf} within the \acs{rpnd}. This may be improved, to some extent, by the
inclusion of a Boltzmann solver. That said, the investigation of the \acs{eedf}s
suggests that the two-term expansion may not be sufficient for the fields in the
\acs{rpnd}. Another approach would be the use of a zero-dimensional \acs{pic}
model, similar to the one used in the \acs{eedf} calculations. It should be
possible develop such a code with the inclusion of the improved cross sections
of Ralchenko et al. \cite{Ralchenko2008} as well as the optical transitions from
Kramida \cite{Kramida2012}, with suitable alterations to account for trapping
effects. This approach would neatly sidestep the \acs{eedf} issues, however
there is the concern that this may lead to an excessively demanding
computational problem. Some simplification could be accomplished using a hybrid
method or by only using \acs{pic} calculations above certain field values.

\section{Final Remarks}

The experimental and simulation results presented represent one of the few
comprehensive attempts to analyze the nanosecond timescale dynamics of a
\acs{rpnd}. Several mechanisms, including the dynamics of the metastable
population and radiation trapping, have been identified which may influence the
evolution and stability of a helium \acs{rpnd}. In addition, evidence has been
provided for beam-like electrons at low operating pressures, and the effects of
impurities on the lifetimes of excited states.

Each of these phenomena represent an important component of how the nanosecond
pulse couples energy into a gas. In turn, it is the energy in that gas--the
electrons, ions, excited states, internal fields, and photons, which lends the
\acs{rpnd} its ability to purify water \cite{Malik2001}, sterilize surfaces
\cite{Ayan2009}, alter air flow \cite{Nishihara2007}, and generate nanoparticles
\cite{Ostrikov2011}. It is only through an improved understanding of how this
energy is distributed in a plasma that applications, such as these, can be
realized.
