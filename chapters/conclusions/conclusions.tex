The repetitively-pulsed nanosecond discharge \acs{rpnd} is a low temperature
plasma with several properties which make it amenable to novel applications.
Some of these properties include its production of a large uniform discharge (on
the order of liters), negligible gas heating, and operation over a wide range of
pressures (from $10^{-3}$--$1$ atmospheres) \cite{Starikovskaia2001}. The
\acs{rpnd} shares a long history with other pulsed discharges including the fast
ionization wave (\acs{fiw}), streamers, sparks, and lightning \cite{Loeb1965}.

Despite the history associated with pulsed discharges, in-depth study of these
phenomena has been challenging. A review of the literature found that the
important dynamics of the \acs{rpnd}, which can occur in a matter of
nanoseconds, have not well-explored. Time-scales this short present a number of
technical difficulties for plasma measurements. Separately, the high electric
fields and the substantial collisionality within the \acs{rpnd} precludes the
use of many traditional plasma diagnostic techniques. Thus, most studies have
focused on measurements of the plasma properties after the pulse or averaged
over the pulse.

However, the properties of the \acs{rpnd} are largely determined by the
phenomena which occur during the onset of the pulse and by the high electric
fields which coincide with it. The \acs{rpnd} can not be understood without a more
detailed consideration of the dynamics which occur during its initial phases.
This includes the transfer of electrical energy to the electron population, as
well as the subsequent excitation and ionization of the background gas.

\section{Overview of Results}

The over-arching goal of the study, then, was to elucidate the flow of energy
from the electric field, through the electrons, to the gas, during the formation
of the \acs{rpnd}. The investigation of the energy flow was subdivided into
several complementary measurements and simulations performed for a helium
\acs{rpnd}. The energy input was characterized through current-voltage
characteristics and estimates of the energy coupling. This was then followed by
measurements of the excited atomic state dynamics with laser absorption
spectroscopy and optical emission spectroscopy. Finally, the results were
compared to estimates produced by a global model of a helium plasma applied to
the \acs{rpnd}.

The process of accomplishing these goals, began with the development of the
theories necessary to describe the \acs{rpnd}. Specifically, the Boltzmann
equation and Maxwell's equations, were used as the starting point of the
theoretical analysis. Several moments of the Boltzmann equation were derived and
presented. These moments included the continuity equation, the conservation of
momentum equation, and the conservation of energy equation. The first and third
of which were later used in the development of global model.

Following the discussion of the statistical descriptions applied to plasmas, the
criteria for a plasma were presented. A plasma is distinct from an ionized gas
in the sense that its behavior is dominated by its electromagnetic properties.
Thereafter, the initiation of a plasma was described for the two cases. The
Townsend mechanism which is traditionally associated with glow discharges and
other long-lived plasmas does not involve the formation of a significant space
charge. In contrast, the streamer mechanism, typically identified by the
Raether-Meek criterion, develops over a short period of time, and is influenced
by the formation of large regions of space charge.

It is the properties of the streamer mechanism which bear the most resemblance
to the \acs{rpnd}. However, the streamer model does not automatically account
for the uniform nature of the \acs{rpnd}. The work of Levatter and Line
\cite{Levatter1980} was used to describe the uniformity which has been observed
in the \acs{fiw} and \acs{rpnd}. Their work demonstrated that a streamer can
develop in a uniform manner provided a sufficient density of pre-pulse
electrons. Calculations for the subsequent experimental conditions found that
the natural background electron density was sufficient to guarantee uniform
breakdown at all operating pressures.

Subsequently, an experimental \acs{rpnd} was described. It was generated by
repetitive positive voltage pulses in ultra-high purity helium at several
different pressures. The discharge geometry was of a coaxial design similar to
the \acs{fiw} studies reviewed by Vasilyak et al. \cite{Vasilyak1994}. Visual
observations and current-voltage characteristics suggested that the plasma was
most stable at a pressure of 4.0 Torr. Estimates of the energy coupling from the
current-voltage measurements indicated that greatest energy coupling occurred at
1.0 Torr, with a peak value of 5.5 mJ. These values were consistent with previous
experimental measurements of the energy coupling in \acs{rpnd}s and \acs{fiw}s.

Laser-absorption spectroscopy (\acs{las}) of the 2$^3$S, or triplet metastable,
state of helium was the primary means by which the nanosecond discharge dynamics
were measured. This technique was chosen as it allowed direct measurement of the
excited states during the onset of the pulse. The \acs{las} yielded measurements
of the temperature and line-integrated density of the triplet metastable state
over the duration of the pulse. The triplet metastable level was chosen because
it can act as a large energy reservoir in helium discharges and can increase the
charged particle population via inter-excited state Penning ionization. The use
of an active optical diagnostic, meant that the time resolution of the
measurements were only limited by the bandwidth of the detector, 5 ns.

The results of the \acs{las} confirmed that no gas heating was occurring in the
\acs{rpnd}. The accuracy of the temperature measurements varied with respect to
the metastable density, but was generally about $\pm50$ K. Some deviation from
room temperature was noted which was possibly associated with inaccurate
pressure broadening coefficients. The largest detected line-integrated
densities, prior to the reflected pulse, occurred for the 4.0 Torr condition at
a value of about 5.9$\times10^{16}$ m$^{-2}$. Assuming a uniform density
distribution across the discharge tube, this is equivalent to a density of
1.8$\times10^{17}$ m$^{-3}$. A detectable number of metastable atoms persisted
between pulses for pressures of 4.0 Torr and lower.

There were no observable trends in the metastable densities with respect to
axial location for pressures of 4.0 Torr and lower. The lack of axial variation
provides additional confirmation of the homogeneous and volume-filling nature of
the \acs{rpnd}. In contrast, results at 8.0 and 16.0 Torr showed a clear decline
in metastables densities as a function of distance from the anode. This behavior
is believed to be akin to the wave attenuation in \acs{fiw}s, described by
Vasilyak et al. \cite{Vasilyak1994}. Similar trends were observed in the
\acs{lcif} data produced by Weatherford et al.\ for the same system
\cite{Weatherford2012}.

Long duration measurements of the metastable densities revealed the primary
decay processes in this discharge geometry. Metastable destruction in the early
post-pulse period was dominated by inter-metastable Penning ionization and
superelastic electron collisions. The importance of these processes was
indicated by deviations from a purely exponential decay. As the pressure of the
system increased and as the time after the pulse increase, the molecular
conversion process became the dominant destruction mechanism for the
metastables. Decay constants significantly larger than those reported by Deloche
et al. \cite{Deloche1976} and Phelps and Molnar \cite{Phelps1953} suggest that
gaseous impurities played a significant role in metastable destruction.

A global model was then developed in order to infer various other plasma
parameters from the metastable measurements. The development of the model
included an analysis of the likely electron energy distribution functions
(\acs{eedf}s) in the \acs{rpnd} with a series of zero-dimensional
particle-in-cell (\acs{pic}) simulations. For fields of 100 Td and less, the
\acs{pic} simulations showed good agreement with Maxwell-Boltzmann distributions
of the same mean electron energy. However, above 300 Td, the Maxwell-Boltzmann
distributions exhibited significantly fewer high-energy (greater than 100 eV)
electrons.

It was speculated that this disagreement stemmed from the behavior of the
interaction cross sections as a function of energy. In general, the electron
interaction cross sections peak at several times the threshold value, after
which they rapidly decline. At high electric fields, electrons can reach this
high energy regime after which they undergo considerably fewer interactions,
resulting in a higher-than-expected population of high-energy electrons. Better
agreement with the \acs{pic} simulations at these field strengths were obtained
with solutions of the two-term expansion of the Boltzmann equation. However, as
a result of notable disagreements at lower electric fields, the global model
assumed a Maxwell-Boltzmann distribution for the entirety of the simulations.

The global model simulations which best matched the measured metastable density
trends required the use of an exceptionally long electric field pulse, 40 ns,
compared to the 25 ns length of the applied pulse. Upon inspection of the
optical emissions from the plasma, a return stroke was identified that appeared
to explain this longer-than-expected excitation. Based on previous results in
\acs{fiw} research \cite{Starikovskaia1998, Starikovskaia2001}, an alternative
explanation was also proposed in which a beam-like electron population formed in
the \acs{rpnd} and its relaxation resulted in the extended excitation.

The global model produced excellent matches to the measured metastable
densities, both during the pulse and afterward. Particularly good agreement was
obtained at 4.0 and 8.0 Torr conditions. Small deviations were observed at 1.0
Torr--the metastable density appeared to rise too quickly during the pulse, and
too slowly after the pulse. These discrepancies also indicated that physics
which were unaccounted for by global model was occurring.

The possibility of energetic electrons was reinforced by the magnitude of the
electric fields and inferred electron temperatures necessary to match the
metastable densities. At 1.0 Torr, the peak electric field was 346 Td and the
peak electron temperature was estimated at 74.6 eV. Though the temperature
estimate is unrealistic for Maxwell-Boltzmann distribution, it may indicate the
presence of beam-like electrons and a slowing-down \acs{eedf} distribution. Such
an \acs{eedf} would be consistent with the large fields observed in \acs{rpnd}s.

In contrast, the global model predictions at 4.0 and 8.0 Torr were better
aligned with previously reported values. The electron temperatures are similar
to those recently obtained for a \acs{fiw} \cite{Takashima2011}, as well as
those predicted from rate coefficient calculations \cite{Aleksandrov2007}.
Likewise, the electric fields are more reasonable than those predicted at 1.0
Torr, though they fall in the intermediate region between 100 and 300 Td.

Interesting to note was a delay between the peak electric field and the highest
ionization rate in the plasma. The delay between the two was interpreted as the
time required for the seed electron population to reach ionization-relevant
temperatures. In a simplified system of population kinetics equations, the
ionization rate is an exponential function of time, thus the density grows
quickly thereafter.

Emission measurements of ten separate transitions in the visible wavelengths
were used to measure the wave velocities of the \acs{rpnd}. While the rate of
rise of the emissions depended on the transition, the wave velocity estimates
remained largely the same. In the \acs{rpnd} the measured velocities were:
1.7-3.0$\times10^7$ m/s at 8.0 Torr, and 0.7-1.5$\times10^7$ at 16.0 Torr. All
other conditions had wave velocities that were greater than 5$\times10^7$ m/s.

The global model was designed to consider a total of 20 different species and
535 different reaction processes, including optical transitions. The inclusion
of all the allowed transitions between excited states made it possible to
evaluate of several potential optical diagnostics for electron temperature and
an opportunity for comparison with results from the measured emissions.

The first attempt to determine the evolution of the electron temperature used
Boltzmann plots generated for each time point. The Boltzmann plots from the
simulated emissions and measured emissions both resulted in temperature
estimates of about 0.5-0.6 eV. However, the temperatures estimated from the
simulated emissions were not consistent with the temperatures produced by the
global mode simulations. This inconsistency indicated that there was likely a
fundamental flaw in the assumptions used to determine electron temperatures from
Boltzmann plots, specifically the assumption of partial local thermodynamic
equilibrium. The inadequacy of this assumption carried well into the afterglow,
up to 4 $\mu$s after the pulse.

Subsequently, an attempt was made to determine the electron temperature via a
coronal model and the use of line ratios \cite{Griem2005}. As with the global
model, this approach requires the assumption of an \acs{eedf}. Thus, the
accuracy of the results will depend on how closely the \acs{eedf} in a
\acs{rpnd} adheres to the assumed one.

For consistency with the global model, a Maxwell-Boltzmann distribution was used
in the development of the coronal model. The use of this distribution likely
produced estimates of the electron temperature that were not accurate with
respect to their absolute magnitude. However, as long as the shape of the
\acs{eedf}s are consistent for the duration of the measurement period, the
results will be proportional to the some mean energy, allowing for an evaluation
and comparison of the relative temperatures in the \acs{rpnd}.

Three line ratios were initially considered however one was immediately rejected
for lack of detectable emissions. The second, which compared the
4$^3$S-2$^3$P$\odd$ transition to the 4$^1$D-2$^1$P$\odd$ transition was used
with little success. The estimates from the simulated emissions were not
consistent with the actual temperatures from the global model, suggesting that
the ratio was not suitable for temperature estimates.

The final line ratio compared the 3$^3$S-2$^3$P$\odd$ transition to the
3$^1$S-2$^1$P$\odd$ transition. In contrast to the previous ratio, the estimates
from the simulated emissions were consistent with the global model temperatures.
This consistency suggests that the coronal model is possibly applicable to this
set of transitions. The estimates from the measured emissions were less
successful. For most points in time, the two transitions were too dim to obtain
a reliable ratio. It was possible to obtain an estimate of the electron
temperature at 4.0 Torr. The results of the line ratio indicated a peak electron
temperature of about 5 eV, followed by a quick cooling to 0.5 eV. As the coronal
model relies on the assumption of a Maxwell-Boltzmann distribution, the use of
such a distribution will almost certainly produce misleading and incorrect
results at pressures of 1.0 Torr and below.

Comparisons were made between the simulated and measured emissions for the
3$^1$D-2$^1$P$\odd$ transition, at 1.0, 4.0, and 8.0 Torr. Some general
features, such as the timing of the peak intensity, and the relative intensities
between pressures were consistent between the two. However the simulated
emissions generally decayed away faster after the peak intensity as compared to
the measured emissions. The relatively long decay periods suggest a prolonged
excitation period in the \acs{rpnd}. Such behavior is consistent with the
extended population of the metastable states at 1.0 Torr, and may be additional
evidence for energetic electrons.

Additionally, the decay rate of the measured emissions had a noticeable pressure
dependence. A pressure dependence, such as this one, suggests the action of an
atomic process one which was not accounted for in the global model. The
metastable measurements suggested that Penning ionization of impurities was an
important loss mechanism. If true, then it is certain that the 3$^1$D state
would also be affected by collisions with the gaseous impurities, leading to a
pressure dependence in the radiative decay rate.

In summary, the presented work identified several key characteristics of the
\acs{rpnd} based on spectroscopic measurements and modeling. Several processes
affecting the charged particle population and energy residence time in the
\acs{rpnd} were identified. In addition, several anomalies were noted at the
analysis of the data at 1.0 Torr and below, indicating a high degree of
non-equilibrium and the potential presence of beam-like electrons. Finally
several approaches to the measurement of electron temperature trends were
evaluated for their application to similar experiments.

\section{Future Work}

There are several opportunities to improve upon the work presented here. From an
experimental perspective, the metastable density measurements could be improved
by the use of a more sensitive detector. The use of such a detector would
decrease the minimum detectable metastable density and potentially provide
pre-pulse metastable densities for the other operating pressures. It would also
be desirable to revisit the existing metastable density measurements along
various chords of the discharge cylinder, which would allow the calculation of
radial density profiles via inversion techniques.

The emission measurements could also be improved by the use of a more efficient
grating, a more efficient photocathode, or both. These changes should be made
with measurement of the 706 and 728 nm lines in mind, as the ratio of the two
appears to be a potential indicator of electron energy in the system. Further
improvements to the electron energy measurements could be made through the
development of an appropriate collisional radiative model and the exploration of
alternative \acs{eedf}s in the calculation of the rate coefficients.

Other experimental work that may be undertaken includes an investigation of the
maximum metastable and electron generation for a fixed pulse-width and varying
electric field. As was described in Chapter~\ref{chp:metastables}, there are
fundamental limits on the rate at which these species can be generated. These
limits may preclude the trend toward shorter pulses which may also limit the
application of the \acs{rpnd} at higher pressures.

Also important to explore further is the effect of electrical reflections and
return strokes on energy deposition in the \acs{rpnd}. Previous studies had
either dismissed or not addressed the importance of these phenomena in
\acs{rpnd}s. However, it was found that both could lead to additional energy
deposition and re-distribution of excited states. A high speed switching system
may offer a practical means by which reflections could be redirected out of the
transmission line. Likewise, careful alteration of the ground-path impedance may
make it possible to alter the nature of the return stroke, or potentially
eliminate it altogether.

The experiment could also be improved with the relatively simple addition of a
capacitive probe for sensing of the electric field within the plasma, similar to
that used by Takashima et al. \cite{Takashima2011}. The electric field
measurements would allow for confirmation of the return stroke which was
observed in the emissions data. The probe may also be used to determine if a
persistent electric field exists after the pulse. Such information would help to
determine the degree to which high energy electrons may influence the discharge.

The electric field information could also be incorporated into the global model
to obtain more accurate predictions of the \acs{rpnd} properties. This
information would eliminate the approximation of the \acs{rpnd} pulse as a
Gaussian. Other improvements to the global model would include the addition of
gas impurities and their associated reactions in response to the evidence that
they contribute to the \acs{rpnd} dynamics.

However, the global model is still fundamentally limited in by the way it
handles the \acs{eedf} within the \acs{rpnd}. The model may be improved, by the
inclusion of a Boltzmann solver. That said, the investigation of the \acs{eedf}s
suggests that the two-term expansion may not be sufficient for the fields in the
\acs{rpnd}. Another possibility is the use of a Boltzmann-like distribution of
the form $f(\epsilon)\propto \exp (\epsilon^\alpha / \kB T_e)$, where $\alpha$
can be varied to obtain a distribution with a suitable high-energy tail.

Alternatively, a zero-dimensional Monte Carlo model could be used, similar to
the \acs{pic} model which was employed in the \acs{eedf} calculations. It should
be possible develop such a code with the inclusion of the improved cross
sections of Ralchenko et al. \cite{Ralchenko2008} as well as the optical
transitions from Kramida \cite{Kramida2012}. Such a model would sidestep the
\acs{eedf} issues, however there is the concern that it may present an
excessively demanding computational problem. Some simplification could be
accomplished using a hybrid method or by only using \acs{pic} calculations above
certain field values.

\section{Final Remarks}

The experimental and simulation results presented represent one of the few
comprehensive attempts to analyze the nanosecond timescale dynamics of a
\acs{rpnd}. Several mechanisms, including the dynamics of the metastable
population, the additional energy deposition of reflected pulses, and the
effects of the return stroke, have been identified which may influence the
evolution and stability of a helium \acs{rpnd}. In addition, evidence has been
provided for beam-like electrons at low operating pressures, and the
presence of gaseous impurities.

Each of these phenomena represent an important component of how the nanosecond
pulse couples energy into a gas. In turn, it is the energy in that gas--the
electrons, ions, excited states, internal fields, and photons, which lends the
\acs{rpnd} its ability to purify water \cite{Malik2001}, sterilize surfaces
\cite{Ayan2009}, alter air flow \cite{Nishihara2007}, and generate nanoparticles
\cite{Ostrikov2011}. It is only through an improved understanding of how the
energy is distributed in a plasma that applications, such as these, can be
realized.
