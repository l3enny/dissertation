The repetitively-pulsed nanosecond discharge \acs{rpnd} is a low temperature
plasma with a several properties which make it amenable to new applications.
Some of these properties include its production of a large uniform discharge, on
the order of liters, negligible gas heating, and operation over a wide range of
pressures \cite{Starikovskaia2001}. Though the \acs{rpnd} has only recently been
enabled by new semiconductor technology, it shares a long history with other
pulsed discharges. This includes the fast ionization wave (\acs{fiw}),
streamers, sparks, and lightning \cite{Loeb1965}.

Despite this long history, in-depth study of these phenomena has been
challenging. A review of the literature found that the important dynamics of the
\acs{rpnd}, which can occur in a matter of nanoseconds, were not well-explored.
This can be attributed to the time scales involved which present a number of
technical difficulties. Separately, the high electric fields and the substantial
collisionality within the \acs{rpnd} precludes the use of many traditional
plasma diagnostic techniques. Thus, most studies have focused on measurements of
the plasma properties after the pulse.

However, as with most plasmas, the properties of the \acs{rpnd} are determined
by the coupling between the applied electric field and the gas. As this coupling
is determined by fundamental processes which occur during the nanosecond pulse,
such as electron acceleration and collisional excitation, the \acs{rpnd} can not
understood without more detailed consideration of the dynamics which occur
during its initial phases.

\section{Overview of Results}

To that end, the necessary theory was developed to describe the \acs{rpnd}. This
work began with the Boltzmann equation and Maxwell's equations, the basic
equations which can be used to describe any ionized gas. Several reductions of
the Boltzmann equation were presented in the form of moments. These moments
averaged over the velocity space of the probability distribution in order to
simplify the solution of the equations.

Following this, the criteria for a plasma were presented. A plasma is distinct
from an ionized gas in the sense that its behavior is dominated by its
electromagnetic properties. Thereafter, the initiation of a plasma from one or a
few electrons was explained for the two primary cases. The first, the Townsend
mechanism, occurs for relatively small electric fields, long periods of time,
and does not involve the formation of appreciable space charge. In contrast, the
streamer mechanism, uses a high electric field, develops over a shorter period
of time, and is influenced by the formation of large regions of space charge.

It is the properties of the streamer mechanism which bear the most resemblance
to the \acs{rpnd}. However, the streamer model did not account for the uniform
nature of the \acs{rpnd}. Instead, this was described by the work of Levatter
and Line \cite{Levatter1980}. They demonstrated that a streamer can develop in a
uniform manner provided a sufficient density of pre-pulse electrons.
Calculations for experimental conditions found that the natural electron density
was sufficient to guarantee uniform breakdown at all operating pressures. The
theoretical background concluded with an explanation of atomic structure,
spectroscopic notation, and spectral lineshapes.

Subsequently, the experimental \acs{rpnd} was described. A positive pulse was
used to initiate the discharge in ultra-high purity helium at several different
pressures. The discharge geometry was of a coaxial design similar to the \acs{fiw}
studies reviewed by Vasilyak et al. \cite{Vasilyak1994}. Visual observations and
the ease with which breakdown was obtained suggested that the plasma was most
stable at a pressure of 4.0 Torr. However, estimates of the energy coupling from
the current-voltage measurements indicated that greatest energy coupling
occurred at 1.0 Torr, with a peak value of 5.5 mJ. This was consistent with
previous experimental measurements of the energy coupling in \acs{rpnd}s and
\acs{fiw}s.

Laser-absorption spectroscopy (\acs{las}) of the 2$^3$S, or triplet metastable,
state of helium was the primary means by which the nanosecond discharge dynamics
were measured. This technique resulted in measurements of the temperature and
line-integrated density of the triplet metastable state over the duration of the
pulse. The triplet metastable level was chosen for several reasons. It is the
lowest excited state in helium and has no allowed decay path to the ground
state. As a result, it can act as an energy reservoir in helium discharges and
can increase the charged particle population via associative ionization and
Penning ionization. These properties made it an obvious choice for study. The
use of an active optical diagnostic, meant that the time resolution of the
measurements were only limited by the bandwidth of the detector, 5 ns.

The results of the \acs{las} confirmed that no gas heating was occurring in the
\acs{rpnd}. The accuracy of the temperature measurements varied with respect to
the metastable density, but was generally about $\pm50$ K. The largest detected
line-integrated densities, prior to the reflected pulse, occurred for the 4.0
Torr condition at a value of about 5.9$\times10^{16}$ m$^{-2}$. Assuming a
uniform density distribution across the discharge tube, this is equivalent to a
density of 1.8$\times10^{17}$ m$^{-3}$. A significant number of metastable atoms
persisted between pulses for pressures of 1.0 Torr and lower. This provides
evidence that the metastable atoms can play an important role in the development
of a helium \acs{rpnd}.

There were no observable trends in the metastable densities with respect to
axial location for pressures of 4.0 Torr and lower. This provides additional
confirmation of the homogeneous and volume-filling nature of the \acs{rpnd}. In
contrast, results at 8.0 and 16.0 Torr showed a clear decline in metastables
densities as a function of distance from the anode. This is believed to be akin
to the wave attenuation in \acs{fiw}s, described by Vasilyak et al.
\cite{Vasilyak1994}.

Long duration measurements of the metastable densities revealed the primary
decay processes in this discharge geometry. Metastable destruction at low
pressures was dominated by associative ionization; indicated by deviations from
a purely exponential decay. As the pressure of the system was increased, the
three-body reaction (leading to the formation of helium dimers) became the
dominant destruction process. Decay constants significantly larger than
those reported by Deloche et al. \cite{Deloche1976} and Phelps and Molnar
\cite{Phelps1953} suggest that gaseous impurities played a significant role in
metastable destruction.

A global model was developed in order to infer various other plasma parameters
from the metastable measurements. The development of the model included an
analysis of the likely electron energy distribution functions (\acs{eedf}s) in
the \acs{rpnd} with a series of zero-dimensional particle-in-cell (\acs{pic})
simulations. For fields of 100 Td and less, the \acs{pic} simulations showed
good agreement with Maxwell-Boltzmann distributions of the same mean electron
energy. At 300 Td and above, the Maxwell-Boltzmann distribution resulted in
significantly fewer high energy (greater than 100 eV) electrons. Better
agreement with the \acs{pic} simulations at these field strengths were obtained
with solutions of the two-term expansion of the Boltzmann equation. However, as
a result of significant disagreements at lower electric fields, the global model
assumed a Maxwell-Boltzmann distribution for the entirety of the simulations.

The global model simulations which best matched the measured metastable density
trends required the use of an exceptionally long electric field--40 ns in
length, compared to the 25 ns length of the applied pulse. Upon inspection of
the optical emissions from the plasma, a return stroke was identified that
appeared to explain this longer-than-expected excitation. It was also
hypothesized that beam-like electron population may form in the \acs{rpnd}
\cite{Starikovskaia1998}, in which case the extended excitation could be a
result of a relaxation by this beam.

Using the long excitation period with the global model produced an excellent
match to the measured metastable densities, both during the pulse and afterward.
Particularly good agreement was obtained at 4.0 and 8.0 Torr conditions. Small
deviations were observed at 1.0 Torr--the metastable density appeared to rise
too quickly during the pulse, and too slowly after the pulse. This provided a
second hint that energetic electrons may be present in the system, specifically
at the lower pressure conditions.

This was reinforced by the magnitude of the electric fields and inferred
electron temperatures necessary to match the metastable densities. At 1.0 Torr,
the electric field was 346 Td and the peak electron temperature was estimated at
74.6 eV. This suggests that the global model is likely inaccurate for this
condition given the disagreement between the Maxwell-Boltzmann results and the
\acs{pic} simulations at values of 300 Td and up. Additionally, the electron
temperature is far in excess of what can be considered reasonable in for an
equilibrium population of electrons in a low temperature plasma.

In contrast, the global model predictions at 4.0 and 8.0 Torr were better
aligned with previously reported values. The electron temperatures are similar
to those recently obtained for a \acs{fiw} \cite{Takashima2011}, as well as
those predicted from rate coefficient calculations \cite{Aleksandrov2007}.
Likewise, the electric fields are more reasonable than those predicted at 1.0
Torr, though they fall in the intermediate region between 100 and 300 Td.

Interesting to note was a delay between the peak electric field and the highest
ionization rate in the plasma. This was interpreted as the time required for the
seed electron population to reach ionization-relevant temperatures. In a
simplified system, the ionization rate is an exponential function of time,
thus the density grows quickly thereafter.

The global model was designed to consider a total of 32 different species and
535 different reaction processes, including optical transitions. This provided
the opportunity to compare simulated plasma emissions generated by the global
model with those measured from the experimental \acs{rpnd}. Emission
measurements of the \acs{rpnd} covered the ten detectable transitions in the
visible spectrum. These were used to estimate the wave velocities of the
\acs{rpnd}: 1.7-3.0$\times10^7$ m/s at 8.0 Torr, and 0.7-1.5$\times10^7$ at 16.0
Torr. All other conditions had wave velocities that were greater than
5$\times10^7$ m/s.

Afterward, an initial attempt was made to determine the electron temperature
from a series of Boltzmann plots. The Boltzmann plots from the simulated
emissions and measured emissions both resulted in temperature estimates of about
0.5-0.6 eV. The temperatures estimated from the simulated emissions showed
significant disagreement with the actual simulated temperatures. This
demonstrated that the use of Boltzmann plots for the \acs{rpnd} is almost
certainly flawed.

The reason for this lies in the assumption of partial local thermodynamic
equilibrium in the plasma. This requires that the excited state populations used
in the Boltzmann plot are in equilibrium with the electrons \cite{Kunze2009}.
However, the \acs{rpnd} likely develops too rapidly for this equilibrium to take
place. This conclusion was reinforced by an examination of the individual
Boltzmann plots for a series of times after the pulse.

Subsequently, an attempt was made to determine the electron temperature via a
coronal model and the use of line ratios \cite{Griem2005}. Three line ratios
were considered however one was immediately rejected for lack of detectable
emissions. The second, which compared the 4$^3$S-2$^3$P$\odd$ transition to the
4$^1$D-2$^1$P$\odd$ transition was used with little success. Estimates based on
the measured emissions produced unrealistic temperatures, and the estimates from
the simulated emissions did not agree with the actual temperatures from the
global model.

The second line ratio which compared the 3$^3$S-2$^3$P$\odd$ transition to the
3$^1$S-2$^1$P$\odd$ transition produced more promising results. The estimates
from the simulated emissions matched the global model temperatures, which
demonstrated that the line could potentially work provided the global model
results are applicable.

However, the estimates from the measured emissions were less successful. For
most points in time, the two transitions were too dim to obtain a reliable
ratio. In the cases that a temperature could be estimated, the results indicated
a peak electron temperature of about 5 eV, followed by a quick cooling to 0.5
eV. As the coronal model relies on the assumption of a Maxwell-Boltzmann
distribution, this approach will almost certainly produce misleading and
incorrect results at pressures of 1.0 Torr and below.

Additional comparisons were made between the simulated and measured emissions
for the 3$^1$D-2$^1$P$\odd$ transition, at 1.0, 4.0, and 8.0 Torr. Some general
features, such as the timing of the peak intensity, and the relative intensities
between pressures were consistent between the two. However, 

\section{Future Work}



\section{Final Remarks}


