The laser-absorption spectroscopy code was written in Python (version 2.7) using
the standard libraries, NumPy (version 1.6), SciPy (version 0.11)
\cite{Jones2001}, and Matplotlib 1.3 \cite{Hunter2007}. The code will be
included here for reference, however the most up-to-date version will be
maintained at \url{https://github.com/l3enny/lasana}. Below is a diagram of the
directory structure used for the laser analysis code.

{
\dirtree{%
  .1 /. 
  .2 analyze.py. 
  .2 atoms. 
  .3 \_\_init\_\_.py. 
  .3 He.py. 
  .2 gui.py. 
  .2 lineshapes.py. 
  .2 main.py. 
  .2 models.py. 
  .2 offset.py. 
  .2 parse.py. 
  .2 preprocess.py. 
  .2 transition.py. 
}
}

The analysis package can be initialized by running the command \texttt{python
main.py} from the root of the directory. This starts a script which handles all
of the settings, submodules, and main processing loop. While much of the program
logic is handled here, the actual calculations are handed off to separate
modules. The code listing for \texttt{main.py} is as follows,
\begin{singlespace}
  \lstinputlisting{./code/lasana/main.py}
\end{singlespace}
It begins by importing several packages from the Python standard library. Theser
are used to manage directory names consistently across different operating
systems. This is followed by the importation of the included submodules and
NumPy.

Also imported at this time is the file defining the physical properties of
helium, \texttt{He.py}.
\begin{singlespace}
  \lstinputlisting{./code/lasana/atoms/He.py}
\end{singlespace}
Includes in this file are the excited atomic states that were considered for the
laser-absorption measurements, their energies, levels, and statistical
degeneracies.

Afterward, the transition class is used to define the helium transitions
of interest to the absorption analysis. The transitions can be held by any
iterator object, in this case a list. The transition class is defined in the
file \texttt{transition.py}.
\begin{singlespace}
  \lstinputlisting{./code/lasana/transition.py}
\end{singlespace}
The class has no built in methods to speak of, instead it is merely acts as a
simple container for the transition information. When defining a transition, it
does make a simple check to ensure that the transition is downward. Of some note
is the fact that it takes the vacuum wavelength and converts it to the air
value.

After the main script defines the transitions, it initializes a graphical user
interface (\acs{gui}) in order to select the directory containing the data to be
analyzed. This functionality is provided by the \acs{gui} submodule contained in
\texttt{gui.py},
\begin{singlespace}
  \lstinputlisting{./code/lasana/gui.py}
\end{singlespace}
The \acs{gui} uses the Tcl/Tk toolkit for interoperability between systems. The
graphical directory picker is contained in the \texttt{pickdir} function which
accepts a string (used to title the window), and a default directory value.
Several other functions are available in the \texttt{gui} submodule which can be
used in analysis, but are not currently employed.

Following the identification of a valid directory, the processing of the data is
handed off to the \texttt{parse.py} submodule.
\begin{singlespace}
  \lstinputlisting{./code/lasana/parse.py}
\end{singlespace}
It contains all of the situation-specific details of where the data are stored,
how to read them in. This submodule can be rewritten to reflect the user's own
particular directory structure, configuration files, file format, etc. In this
case when the \texttt{config} function is called, the standard module,
\texttt{ConfigParser}, is used to read in processing information from a
configuration file generated by the LabView data acquisition software. This is
returns a dictionary with all of the configuration settings to the main script.

Subsequently, the main script calls the \texttt{data} function from the
\texttt{parse} submodule in order to load the actual experimental data sets.
This function takes a directory as its input as well as the settings. Also
included in the function is the laser-specific conversion value from current to
frequency. The function returns the data set, the acquisition time array, and
the frequency values for which the data were obtained. These are returned to the
main script which then loads the data from the background.

These two data sets are then passed to the preprocessor which is defined in
\texttt{preprocess.py}.
\begin{singlespace}
  \lstinputlisting{./code/lasana/preprocess.py}
\end{singlespace}
The preprocessor makes the necessary adjustments, noise subtractions, and laser
power corrections, in order to obtain the transmission spectra for each time
step. The transmission spectra are then passed back to the main script for
analysis.

At this point, the main script defines the model to which the transmission
spectra will be matched. This is defined in the models submodule,
\texttt{model.py}.
\begin{singlespace}
  \lstinputlisting{./code/lasana/models.py}
\end{singlespace}
Several models are available in this submodule including a Gaussian model and a
bi-modal Voigt. None of these functions are the actual lineshape, but rather the
lineshape multiplied by the necessary constants to obtain the cross sections,
followed by exponentiation for the actual transmission curve. The definition of
the Voigt, Lorentzian, Gaussian, and pseudo-Voigt profiles is done in
\texttt{lineshapes.py}.
\begin{singlespace}
  \lstinputlisting{./code/lasana/lineshapes.py}
\end{singlespace}

After the main script has defined the model, it generates reasonable starting
guesses to seed the solver. This is followed by the allocation of data arrays
for the converged solutions and the covariance matrices. Finally, the script
hands off the model and the transmission spectra to the analysis submodule,
\texttt{analyze.py}.
\begin{singlespace}
  \lstinputlisting{./code/lasana/analyze.py}
\end{singlespace}
At one point, this submodule contained its own matching algorithm. However, this
was eventually replaced with SciPy's \texttt{curve\_fit} function which utilizes
the Levenberg-Marquardt algorithm.

Once the main script has successfully looped over all the transmission spectra,
it has finished. A fair amount of post-processing is included in the present
version however no further data analysis is done. One additional file is
included in the laser-absorption analysis package. This is the
\texttt{offset.py} script.
\begin{singlespace}
  \lstinputlisting{./code/lasana/offset.py}
\end{singlespace}
This script is functionally the same as the \texttt{main.py} script, with the
exception that its sole purpose is to determine the frequency offset which is
used in the configuration file. This script was written when it became clear
that the diode laser frequency drifted by tens of MHz during the course of
regular operation. Briefly, the script determines the transmission spectra with
the largest signal and then uses a modified modeling function to calculate its
offset from zero.

