\section{Overview}

The study of atmospheric-pressure plasmas (\acs{app}s) is indistinguishable from
the subject of plasmas as a whole. However, the detail of the measurements and
calculations have historically been limited by the complexity of \acs{app}s. The
high pressure generally equates to a large number of neutral collisions which
can drastically increase the number of possible reactions and the computational
difficulty. In many cases, the data to model anything but the simplest reactions
does not exist. Likewise, the high number of collisions can significantly
complicate data analysis for a great number of plasma diagnostics. In addition,
\acs{app}s typically feature high electric fields, short time scales, and
unpredictable shapes, all of which make even the most basic observations
challenging.

In the last several decades, some of this has begun to change. The advent of
high-powered computing has produced a number of simulations for \acs{app}s with
models of previously inaccessible detail. Similarly, advances in plasma
diagnostics (particularly those based on optical phenomena) has made precise
measurements in these extreme states of matter possible. As a result, the body
of knowledge associated with \acs{app}s--dominant chemical reactions, degree of
gas heating, modes of relaxation, etc.--has increased with tremendous speed.
While some of this work is driven solely by scientific curiosity, much of it can
be attributed to the broad range of applications for \acs{app}s.

Indeed, the applications are almost too numerous to list. One of the first uses
of an \acs{app}, and plasma general, was to generate ozone for the purification
of water. Around the same time, arc lamps (another type of \acs{app}) became a
popular means for public lighting. To this day, similar arcs are the dominant
industrial means for cutting and welding metals. In a sense, \acs{app}s have
become almost mundane; the polymer skin of almost every potato chip bag is
plasma-treated in order to improve the adsorption of ink.

Both of these cases represent relatively low-value transformations. For example,
the cost of the plasma-treated polymer used for potato chip bags is on the order
of cents. However, researchers have identified a number of opportunities where
\acs{app} treatment can produce high-value products. One such example is the use
of plasmas as actuators on airfoils. An \acs{app} can alter the flow of air over
a surface, and significantly increase the flexibility of modern aircraft. A
surprisingly similar plasma can be used to sterilize tissue wounds and treat
infections. Other work has shown that \acs{app}s can improve fuel efficiency in
combustion engines.

In each of these cases, the way the plasma is formed, as well as its
characteristics, can be immensely different. The original arc discharges were
created between graphite rods connected to immense battery banks. Meanwhile, a
modern research reactor studying plasma-assisted combustion might use a
fast-switching semiconductor circuit. Over the years, several types of
\acs{app}s have been developed for a number of situations: dielectric-barrier,
corona, thermal arc, RF, microwave, pulsed, and more.

Within this array\footnote{The interested reader is referred to Starikovskaia's
review \cite{Starikovskaia2006} which provides a general overview of \acs{app}s
in the context of plasma-assisted combustion}, the repetitively-pulsed
nanosecond discharge (\acs{rpnd}) is particularly interesting. Generally
speaking, a \acs{rpnd} is generated by an electrical pulse on the order of $>1$
kV, lasting anywhere from $<1$--$100$ ns. This pulse generates a wave of
ionization (and light) which crosses from the powered electrode to the grounded
one.

The resulting plasma can fill large volumes (on the order of liters) and can
excite many atomic and molecular states with little heating. Changes to the
pulse shape and amplitude can be used to enhance specific atomic and/or
molecular reactions. This flexibility is highly desirable in plasma processing
where the emphasis is on selectivity of the reaction, and control of its rate.

\acs{rpnd}s have been the extensively studied by several different groups of
researchers. However, the body of work is focused on \acs{rpnd}s in air. 

Several research groups have focused on the development of pulsed nanosecond
discharges in air. However, the complex chemistry associated with air plasmas
obscures some of the more fundamental questions: how is the pulsed nanosecond
plasma formed, how are the excited states of the system populated, what
significance is there to reactions after the fast ionization wave, the spatial
variation of system parameters, what kind of electron energy distribution can be
expected? This study will emphasize spectroscopic measurements of a helium
pulsed nanosecond discharge.

Such a system retains physical relevance given that helium is a common
stabilizing additive to atmospheric pressure plasmas. At the same time, the more
simple atomic structure lends itself to a more detailed examination using global
models, kinetic simulations, and active spectroscopy. This work will focus its
efforts on the description of the spatial variation of the pulsed-nanosecond
discharge, an approximation of its electron energy distribution function, and a
description of how the neutral atoms are excited.

The remainder of this chapter includes a review of the associated literature, as
well as a discussion of basic discharge theory (diagnostic-specific theory will
accompany the relevant chapters). Chapter \ref{chp:exp} provides the necessary
details of the experimental setup as well as some preliminary observations.
Next, in chapter \ref{chp:meta}, absolute measurements of the helium triplet
metastable densities are presented. Chapter \ref{chp:emit} covers more general
emissions measurements. In order to provide a more clear understanding of the
meaning of these measurements, we explore them using a global model and a
particle kinetics model. Finally, we conclude in chapter \ref{chp:conc} with a
discussion of how the collection of models and measurements influences our
understanding of the pulsed nanosecond discharge.

\section{Literature Review}

\subsection{History of Atmospheric-Pressure Discharges}

Like most physical phenomena, plasmas are typically only described under ideal
circumstances. This means that neutral collisions, and subsequently, atmospheric
plasmas, are often ignored. Neutral collisions tend to obscure the
electromagnetic effects that distinguish a plasma from a gas. However, the
history of observation and study of plasmas is indelibly linked to atmospheric
plasmas. Lightning and static sparks are the most prevalent plasmas on earth. 
Indeed, the among the first artificial plasmas was a constant atmospheric arc
generated in 1802, the work of a Russian scientist named Vasilii Petrov
\cite{Anders2003}.

Indeed, at present, there's a relative abundance of different atmospheric
pressure discharges: dielectric-barrier, RF, microwave, glow, pulsed, and more.
Each has unique range of parameters in which it operates, and a survey of all
possible atmospheric discharges would be excessive. Instead, we shall focus on
the two most common discharges which, together, illustrate the distinct
advantages of the nanosecond-pulsed discharge.

The work of Petrov was the forerunner to what is now referred to as the study of
thermal arcs. Such plasmas are common in industrial lighting systems, and for a
time were even used as the primary nigh-time illumination of Detroit. Perhaps
more common, is the use of thermal arcs for the welding or cutting of metals.
This potential was recognized early on. Volta's recent invention of the voltaic
pile provided the first source of constant and sufficient electrical energy.
Using a series of voltaic cells, Petrov was able to draw the first electrical
arc between two sticks of carbon. Its blinding light was recorded in a number of
historical prints, such as figure \ref{fig:humphry}.
\begin{figure}\label{fig:humphry}
  \centering
  \includegraphics[scale=0.25]{chapters/introduction/figures/humphry.jpg}
  \caption{This is a test.}
\end{figure}
Aside from this light, the arcs were characterized by their significant
ionization, and high degree of thermal equilibrium. Gas temperatures could reach
thousands of kelvin.

Both the primary advantage and disadvantage of thermal arcs are their high
operating temperatures. These high operating temperatures correlate with the
ability of the plasma to convert electrical energy to thermal energy; of great
values in welding or for continuum light generation. In these plasmas, some of
the energy is also converted to excited atoms or molecules. In some cases, these
species may be desirable for processing a material. However, contact with these
high temperatures will often lead to destruction of the substrate.

In contrast, later work by Werner von Siemens \cite{Siemens1857}, led to the
discovery of the so-called ``silent discharge,'' seen in figure
\ref{fig:siemens}.
\begin{figure}\label{fig:siemens}
  \centering
  \includegraphics[scale=0.5]{chapters/introduction/figures/siemens.png}
  \caption{Werner von Siemens' silent discharge. Per Rackham requirements, the
caption text should be singled-spaced, regardless of the text body. I'm giving
this a test right here. Yep. It should stay single-spaced. Pretty please?}
\end{figure}
In recent years, the terminology has changed and this type of
discharge is now referred to as a dielectric-barrier discharge, or \acs{dbd}.
The \acs{dbd} was significantly different from the thermal arc. Visually, it was
much dimmer, and appeared to be composed of many thousands of individual
filaments. Additionally, the \acs{dbd} did not significantly heat the air,
unlike the thermal arc. Finally, the \acs{dbd} was used in the first commercial
plasma application: ozone generation and water purification. Notably, both the
thermal arc and silent discharge predated the `official' discover of plasma by
Sir William Crookes in 1872.

The \acs{dbd} could be used process materials. Indeed, it has become common to
use this type of discharge to treat polymer films, as well as clothes, and
medical equipment (sterilization). However, like the thermal arc, there are some
limitations. \acs{dbd}s are not particularly uniform as a result of the large
numbers of filaments. Thus, uniform processing will only occur as a result of
long processing times--when the relatively random position of these filaments is
averaged out--or if the desirable species has a long enough life to diffuse away
from where it was created. Another limitation to \acs{dbd}s is that the
filaments characteristics are not a strong function of the applied voltage. This
leads to limited flexibility in control of the plasma characteristics.

\subsection{Nanosecond-Pulsed Plasmas}

For a substantial period of time, these two discharges represented the range of
atmospheric-pressure plasmas (\acs{app}). The thermal arc, though useful, could
not be used on delicate substrates. It had the additional problem of having
relatively little control over its chemical kinetics. Meanwhile, the \acs{dbd}
was relegated to ozone production and polymer processing (relatively low-value
applications). Though the \acs{dbd} had attractive thermal properties, little
else was known about how it operated, and how to control its properties. As
recently as 2007, the National Academies noted that ``the full promise of
\acs{app}s will be known only if they can be understood and managed based on
fundamental scientific principles at two extremes--the nanoscopic kinetic level,
where selective chemistry occurs, and the global stability level, likened to
aerodynamics.'' \cite{NA2007}

As with the discovery of the thermal arc and the dielectric barrier discharge,
the first pulsed discharges were studied prior to the official discovery of
plasma. In 1835, Charles Wheatstone attempted to measure what he called the
velocity of electricity in a spark gap, powered by a Leyden jar
\cite{Wheatstone1835}.
\begin{figure}
  \centering
  \includegraphics[width=4in]{chapters/introduction/figures/wheatstone.png}
  \caption{Sketches of the discharge apparatus and measurement system used by
Wheatstone.}
\end{figure}
In hindsight, the question was poorly phrased; Maxwell's
equations had not yet been formalized, and electrons weren't recognized as the
carriers of electrical current. Nevertheless, though poorly controlled, the
phenomena was essentially the same as modern pulsed nanosecond discharges.

Conceptually, the work of Wheatstone was promising, however the measurements
were quite inaccurate. One particularly important outcome of Wheatstone's work
was the subsequent observation by von Zahn \cite{Zahn1879} that the particles
emitting the light were not travelling at anywhere near the velocity of the wave
of light. J.J. Thomson later repeated the experiment with a 15 m tube, in order
to obtain an improved estimate of the wave velocity and its direction
\cite{Thomson1893}. He estimated the velocity of the wave to be approximately
one half the speed of light, traveling from the anode to the cathode.

These high velocities caused an moderate amount confusion, particularly because
there appeared to be no associated motion of the emitting particles. Subsequent
examinations be Beams \cite{Beams1930} confirmed the velocity measurements, and
more importantly demonstrated that, regardless of the polarity of the applied
potential, the luminous wave always travelled from the high potential to the low
potential electrode. He additionally noted that the large pulse of current
associated with such waves was not detected until after the wave had crossed the
length of the tube. This observation led to the astute observation by Beams that
the apparent motion of the luminous front was more likely a moving region of
ionization.

As observed by Loeb \cite{Loeb1965} in his unifying description of pulsed
nanosecond discharges, aside from the group of people studying the propagation
of light in rarefied gas discharges, were a different set of scientists working
on the origin of lightning. The endeavor was ambitious, for many of the same
reasons that atmospheric discharges have always been difficult to study.
Basically, no one was quite sure where to point their cameras, and when to open
the shutters. Though out of the scope of this paper, the information gleaned
from these studies was both relevant and useful in the development of the theory
underlying ionization waves. Interested readers are referred to the review by
Gurevich and Zybin \cite{Gurevich2001} for an overview of the models employed in
describing natural lightning.

Though studies continued intermittently after a burst of interest in the 1930s,
it wasn't until the mid-1960's and 1970s when any significant attention came to
the pulsed-nanosecond discharge. Beginning with Loeb's description of the
phenomena as ``ionizing waves of potential gradient,'' several more articles
appeared exploring the nature of these pulses. Mesyats, Byckhov, and Kremnev
\cite{Mesyats1972} developed a theory explaining the breakdown process which is
among the first to specifically reference runaway or nonlocal electrons. This
theory was continued with a kinetic treatment and several simulations by
Kunhardt \cite{Kunhardt1980}, and Kunhardt and Tzeng \cite{Kunhardt1988}.



\subsection{Research Plan}

Propose research to fill this gap
Specific and cited history of PNDS and related measurements.

\section{Basic Theory}

Basic theory of gaseous breakdown.
