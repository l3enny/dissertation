\section{Background}
It is difficult to definitely define the discovery of plasma. The observation of
plasmas has spanned the whole of human existence, from the very moment we began
to gaze at the stars. Wikipedia credits Sir William Crookes with first
identifying it as a new state of matter in $1872$ (though Crookes credits
Faraday with hypothesizing its existence as early as $1819$). Curiously, the
first applications of plasma preceded its discovery as a new state of matter by
$15$ years when Ernst Werner von Siemens described the use of a ``silent
discharge'' in air for the preparation of ozone in the treatment of biologically
contaminated water.

To this day, studies continue to explore the use of plasma to treat water, now
with concern for industrial pollutants. In general, the last decade has carried
a surge of research on the development and application of atmospheric-pressure
plasmas. However, the formation of such plasmas has always been shrouded in a
degree of mystery. Most plasma diagnostics are ill-suited for use with
atmospheric-pressure discharges, usually because they are too slow, make
untenable assumptions, or both. Only recently has the technology and techniques
reached a point at which we can begin to understand what is really happening in
these plasmas.

My work concerns the study of one such atmospheric-pressure plasma. In the
literature, it goes by several names; here, I will simply refer to it as a
repetitively-pulsed nanosecond discharge (\textsc{rpnd}). Of course, the choice
of a name is only moderately useful, more important is a description. The
\textsc{rpnd} is a plasma that exhibits exceptional uniformity and volume, with
minimal gas heating.

\section{Theory}

\section{Literature Review}

