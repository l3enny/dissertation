\section{Overview}

\todo[inline]{Answer this question: how many references do you want here? How
many are you going to delay until the lit review?}

\todo[inline]{Short paragraph, non-rigorous definition of plasmas. Be sure to
  clarify distinction from electrical discharge, or if you consider them the
same.}

Historically, the study of atmospheric-pressure plasmas (\acs{app}'s) is
indistinguishable from the study of plasmas as a whole. However, the detail of
the measurements and calculations associated with \acs{app}'s has been limited
by their complexity. From a computational perspective, the high pressure and
number of potential reactions present a difficult challenge. Likewise, the high
pressure can significantly complicate the data analysis for a number of plasma
diagnostics. Aside from the high pressures, the large electric fields, short
time scales, and general randomness of \acs{app}'s make even the most basic
observations a feat.

\todo[inline]{Ambiguous start to paragraph, specify and cite problems, then
identify how they have been overcome.}

In the last several decades, some of this has begun to change. High-powered
computing has allowed simulations with remarkable detail. Similarly, advances in
technology has enabled plasma diagnostics in regimes that were experimentally
inaccessible. As a result, the body of knowledge regarding \acs{app}'s has
greatly increased. Sometimes, the motivation for this work is scientific
curiosity. More often, the study of \acs{app}'s has been driven by a broad range
of applications.

Among the first plasma applications were provided by \acs{app}'s: ozone
generation and lighting. Aside from these items, plasma welding, polymer
treatment, combustion, and plasma televisions have become widely accepted.
Meanwhile, a large number of new applications may soon be added to this list,
including: treatment of tissue wounds, altering airflow over airfoils, and
destruction of industrial pollutants.

Unsurprisingly, each case demands a different kind of plasma. The original arc
discharges were created between two graphite rods connected to immense battery
banks. In contrast, a modern research reactor studying plasma-assisted
combustion might use a fast-switching semiconductor circuit. Over the years,
several types of \acs{app}s have been developed for a variety of situations:
dielectric-barrier, corona, thermal arc, RF, microwave, pulsed, and more.

Within this group\footnote{The interested reader is referred to Starikovskaia's
review \cite{Starikovskaia2006} which provides a general overview of \acs{app}'s
in the context of plasma-assisted combustion}, the repetitively-pulsed
nanosecond discharge (\acs{rpnd}) has created considerable interest. Generally
speaking, a \acs{rpnd} is a plasma generated by a repetitive electrical pulse
applied between two electrodes. The pulse voltage is often in excess of one
kilovolt, lasts anywhere from $<1-100$ ns, and is repeated over a thousand
times each second. The result is a wave of ionization (and light) which crosses
from the powered electrode to the grounded one.

A \acs{rpnd} can fill volumes of several liters with a relatively uniform
plasma. Though they can cause significant excitation of the background gas, they
generally produce very little heating (in some cases below a detection limit of
$\Delta \pm 15$ K). In addition, the excitation can be changed with adjustments
to the magnitude or duration of the electrical pulse. Each of these
characteristics are highly desirable in one or more of potential applications
for \acs{app}'s.

Given all of these promising properties, \acs{rpnd}'s have been the subject of
substantial study by several research groups. However, much of this work has
focused on the physics of \acs{rpnd}'s in air. Unfortunately, air's large number
of constituent elements can lead to notable complexity. In turn, this can
obscure some of the more fundamental questions relation to \acs{rpnd}'s: how do
they form, how is the energy distributed between excited particles, and what
kind of spatial variation can be expected?

This paper details a study of each of these questions in a helium \acs{rpnd}.
Specifically, the densities of one particular excited atom are measured for a
variety of pressures and locations. This is complemented by measurements of the
light emissions for the same set of parameters. A simple model of a \acs{rpnd}
is used to predict several characteristics of the plasma based on the excited
state densities: electron density, electric field, and light emission. The
measured light emissions are interpreted to show how the energy is distributed
in the gas, and how it changes over time. Finally, they are compared with the
estimated light emissions to check the validity of several common assumptions.

\todo[inline]{Paragraph on reason for choosing helium, specify choice of
measurements and topic of dissertation}

The remainder of this chapter is comprised of a review of the associated
literature, as well as a discussion of basic discharge theory. Chapter
\ref{chp:exp} covers the experimental setup as well as some general observations
of the \acs{rpnd}. Next, the measurement of the excited state densities is
presented, followed by the chapter on the light emission measurements. Chapter
\ref{chp:model} explores the global model used to interpret the excited state
densities, as well as some supporting particle-in-cell simulations. Finally, the
paper concludes with a discussion of how the models and measurements impact the
present understanding of \acs{rpnd}'s.

\section{Literature Review}

Though \acs{rpnd}'s are very much a product of twentieth century research, they
are fundamentally similar to a number of other pulsed discharges such as
electrical sparks and lightning. Though Loeb united these disparate fields under
the title of ``ionizing waves of potential gradient'' in 1964 \cite{Loeb1965}
(we use the more familiar term, fast ionization waves), the underlying subjects
had been under study since the Greeks who generated sparks by rubbing together
amber and fur.

Despite these early observations, it was Leibniz in 1671 who first came to the
conclusion that sparks were an electrical phenomena \cite{Kryzhanovsky1989}.
Subsequently, Franklin's famous kite experiment led him to a similar conclusion
on the nature of lightning. Franklin was also involved in explaining the
principles of Leyden jars, develop by Musschenbroek. The Leyden jar was the
first reliable way to store electrical energy and proved a boon to later
research.

In 1835, Wheatstone made the first attempt to measure the speed of electricity
through a gas \cite{Wheatstone1835}. In his work, Wheatstone used a Leyden jar
connected to two metal spheres, separated by a small gap. Once the charge in the
jar reached a critical level, a spark would form in the gap. Figure
\ref{fig:wheatstone} shows the experimental sketch provided by Wheatstone.
\begin{figure}
  \centering
  \includegraphics[width=4in]{chapters/introduction/figures/wheatstone.png}
  \caption{The experimental sketches of Wheatstone showing a traditional
  spark gap connected to a Leyden jar and electrostatic
generator.}\label{fig:wheatstone}
\end{figure}
Though the measurement is notable for its early date, it was later revisted with
much more accuracy by Thomson \cite{Thomson1893}. Perhaps the most important
outcome of Wheatstone's study was the observation by Zahn \cite{Zahn1879} that
the speed of the light was \emph{not} accompanied by a similar motion of the
emitting particles.

Thomson's work concerned both the speed and direction of light in a pulsed
discharge. Unlike Wheatstone's study, Thomson used an elongated tube, 15 m in
length, and 5 mm in diameter, upon which he drew a vacuum. The original sketch
of Thomson's discharge apparatus can be seen in figure \ref{fig:thomson}.
\begin{figure}
  \centering
  \includegraphics[width=4in]{chapters/introduction/figures/thomson.png}
  \caption{A sketch of J.J. Thomson's early experiments on fast ionization
  waves in long vacuum tubes.}\label{fig:thomson}
\end{figure}
Through a clever arrangement of mirrors, Thomson determined that the electricity
had a speed approaching $1\times10^{10}$ cm/s, and travelled from the anode to
the cathode.

It was later, in 1930, that Beams would determine that the wave always initiated
at the high voltage electrode, regardless of polarity \cite{Beams1930}. In
addition, Beams measured the current at the low potential electrode. He detected
a current pulse which did not appear until after the light had completely
crossed the gap. He came to the conclusion that the luminous front was likely
the result of a moving region of ionization.

Around the same time, there was a distinct set of researchers ho were studying
similar phenomena in lightning. In most cases, these studies concentrated on
time-resolved photography, pioneered by Boys\footnote{In the same article, Boys
anticipated a number of other atmospheric physics studies by proposing that
rockets be fired at thunderclouds. Unfortunately, he lived in a village of
thatched houses and could not conduct the experiment for fear of
fire.}\cite{Boys1926}, and refined by Schonland \cite{Schonland1935}. This
technique was later adopted by Allibone and Meek \cite{Allibone1938} to observe
the evolution of a laboratory-generated spark.

By 1935, fast ionization waves had been under study for nearly 50 years.
However, there was still no adequate explanation for the speed of the discharge.
Similarly, Beams' observation that the wave always travelled from the high
voltage to the low voltage electrode (regardless of polarity) could not be
accounted for. Based on observations made with fast pulses, Flegler and Raether
developed a new theory of breakdown for sparks in air \cite{Flegler1936}
which was capable of, at least partly, explaining the fast ionization wave
phenomena. Independently, Loeb and Meek developed a similar theory in 1940
\cite{Loeb1940}.

\todo[inline]{Remember to include the laser dudes! Xenon lamps!}

\section{Basic Theory}

\subsection{Plasmas}

A volume containing some number of charged particles can be considered a plasma
if it meets three conditions. The first requires that the motion of charged
particles is primarily determined by the electric and magnetic fields of the
volume rather than through collisions with neutral particles. This is
classically expressed by the inequality
\begin{equation}
  \sqrt{n_\mathrm{e} e^2 / (\epsilon_0 m_\mathrm{e})} < \nu,
\end{equation}
where $n_\mathrm{e}$ is the electron density, $e$ is the fundamental charge,
$\epsilon_0$ is the permittivity of free space, $m_\mathrm{e}$ is the mass of an
electron, and $\nu$ is the electron-neutral collision frequency. The left-hand
side term is called the electron plasma frequency, it the characteristic
frequency at which a plasma oscillates in response to a perturbation.

\todo[inline]{requirements for a plasma}

For a sufficiently large number of particles, the behavior of the each species
of the plasma can be described by a continuous probability distribution
function. This function, $f_\alpha(\vec{r}, \vec{v}, t)$, describes the
probability of finding a particle of species $\alpha$, at position $\vec{r}$,
The distribution function for a particle can be determined by the
Vlasov-Fokker-Planck \acs{vfp} equation,
\begin{equation}\label{eq:vfp}
  \frac{\partial f_\alpha}{\partial t} + \vec{v_\alpha}\cdot\nabla f_\alpha +
  q_\alpha \left(\vec{E} + \vec{v_\alpha}\times\vec{B}\right)
  \cdot \nabla_\mathrm{v} f_\alpha = \left( \frac{\partial f_\alpha}
  {\partial t}\right)_\mathrm{coll}.
\end{equation}
Here, $\vec{E}$ is the electric field, $\vec{B}$ is the magnetic field, and
$\partial f_\alpha/(\partial t)_\mathrm{coll}$ is a term representing all
collisions. The \acs{vfp} equation is coupled to Maxwell's equations in order to
obtain a self-consistent description of the particle distribution and the
resulting fields. In essence, this is the Boltzmann equation from statistical
mechanics, however it now includes several changes. Vlasov replaced the original
force term with the Lorentz equation, and Fokker and Planck introduced the
collision operator on the right-hand side. This is coupled with Maxwell's
equations for a solution of the electric and magnetic fields in the plasma.

In the absence of external fields and with only elastic collisions, the equation
admits the famous Maxwell-Boltzmann equilibrium distribution,
\todo[inline]{n for degrees of freedom? Change to something unambiguous?}
\begin{equation}\label{eq:mb}
  f_\alpha(v) = n\left(\frac{m_\alpha}{2\pi k_\mathrm{B}T}\right)^{3/2}
    \exp\left(-\frac{m_\alpha v_\alpha^2}{2k_\mathrm{B}T}\right),
\end{equation}
where $n$ is the number of degrees of freedom, $k_\mathrm{B}$ is Boltzmann's
constant, and $T$ is the temperature. A species of particles which possesses a
Boltzmann distribution is said to be in equilibrium. Likewise, two species with
the same distribution are in equilibrium.

\todo[inline]{Should the Boltzmann-Maxwell equilibrium be pushed back to
discussion of rate equations and averaging?}

Aside from this, the \acs{vfp} equation is notoriously difficult to solve. As a
result, most plasma models use various moments of equation \ref{vfp} where the
velocity dependence has been integrated out. For example, the first moment is
the continuity equation,
\begin{equation}\label{eq:cont}
  \frac{\partial n_\alpha}{\partial t} + \nabla \cdot (n_\alpha \vec{u_\alpha})
  = G_\alpha - L_\alpha,
\end{equation}
where $\vec{u_\alpha}$ is the mean velocity of species $\alpha$, $G_\alpha$ is its
rate of gain, and $L_\alpha$ is the rate of loss. This equation can be
interpreted as the rate of change in particle density for a particular volume of
space. 

Though the continuity equation is much simpler than the original \acs{vfp}
equation, it cannot be solved alone. The mean velocity, $\vec{u}$, is undefined.
Typically, this leads to the second moment, \todo[inline]{Come up with tensor
notation, and fix collision operator}
\begin{equation}\label{eq:mom}
  mn_\alpha\left[\frac{\partial \vec{u_\alpha}}{\partial t}
  (\vec{u_\alpha}\cdot \nabla)\right] = q_\alpha n_\alpha(\vec{E} +
  \vec{u_\alpha} \times \vec{B}) - \nabla \cdot \vec{\Pi_\alpha} +
  \vec{f}_{\alpha,c}
\end{equation}
where \vec{\Pi_\alpha} is the pressure tensor, and $\vec{f}_{\alpha,c}$ is the
rate of momentum transfer into species $\alpha$. Again, any solution is stymied
by the presence of a new a term, in this case, $\vec{\Pi_\alpha}$. At this
point, an equation of state can be used to close the set of equations, in this
case relating the pressure to the density. However, later work will benefit from
one more moment.

Following the conservation of momentum, the energy conservation equation can be
derived from the \acs{vfp} equation,
\begin{equation}
  \frac{\partial}{\partial t}\left(\frac{3}{2}p\right)
\end{equation}

\subsection{Fast Ionization Waves}

\subsection{Atomic Spectroscopy}


