\section{Overview}

Research on atmospheric plasmas has been going on for as long as research on
plasmas themselves. However, it was only recently that the necessary diagnostic
techniques and computational power existed to properly examine them. In the last
few decades there has been a renaissance in work dedicated to atmospheric
plasmas, particularly atmospheric pressure plasmas that are out of thermal
equilibrium. This is because they are able to treat delicate substrates with
none of the thermal damage associated with arcs.

Though several approaches exist to generating atmospheric pressure plasmas (dbd,
microwave, rf), particularly interesting are those using nanosecond pulses. The
duration of such pulses, as well as their amplitude can be changed to target
particular atomic or molecular reactions. This flexibility is highly desirable
in the world of plasma processing where selectivity, control, etc. are of utmost
importance. However, we are still learning how to measure these particular
plasmas. As recently as $1994$ \todo{cite Vasilyak} the only diagnostic with any
meaningful accuracy was propagation velocity. 

Several research groups have focused on the development of pulsed nanosecond
discharges in air. However, the complex chemistry associated with air plasmas
obscures some of the more fundamental questions: how is the pulsed nanosecond
plasma formed, how are the excited states of the system populated, what
significance is there to reactions after the fast ionization wave, the spatial
variation of system parameters, what kind of electron energy distribution can be
expected? \todo{simplify for general reader} This study will emphasize
spectroscopic measurements of a helium pulsed nanosecond discharge.

Such a system retains physical relevance given that helium is a common
stabilizing additive to atmospheric pressure plasmas. At the same time, the more
simple atomic structure lends itself to a more detailed examination using global
models, kinetic simulations, and active spectroscopy. This work will focus its
efforts on the description of the spatial variation of the pulsed-nanosecond
discharge, an approximation of its electron energy distribution function, and a
description of how the neutral atoms are excited.

\section{Literature Review}

\subsection{History of Atmospheric-Pressure Discharges}

Like most physical phenomena, plasmas are typically only described under ideal
circumstances. This means that neutral collisions, and subsequently, atmospheric
plasmas, are often ignored. Neutral collisions tend to obscure the
electromagnetic effects that distinguish a plasma from a gas. However, the
history of observation and study of plasmas is indelibly linked to atmospheric
plasmas. Lightning and static sparks are the most prevalent plasmas on earth.
Indeed, the first artificial plasma was an atmospheric arc, the work of a
Russian scientist named Vasilii Petrov.

The work of Petrov was the forerunner to the study of thermal plasmas. In 1802,
Volta's recent invention of the voltaic pile provided the first source of
constant electrical energy. Using a series of voltaic cells, Petrov was able to
draw the first electrical arc between two sticks of carbon. Aside from its
blinding light, these arcs were characterized by their significant ionization,
and high degree of thermal equilibrium. Gas temperatures could reach thousands
of kelvin.

In contrast, later work by Werner von Siemens, led to the discovery of the
so-called ``silent discharge.'' In recent years, the terminology has changed and
this type of discharge is now referred to as a dielectric-barrier discharge, or
\acs{dbd}. The \acs{dbd} was significantly different from the thermal arc.
Visually, it was much dimmer, and appeared to be composed of many thousands of
individual filaments. Additionally, the \acs{dbd} did not significantly heat the
air, unlike the thermal arc. Finally, the \acs{dbd} was used in the first
commercial plasma application: ozone generation and water purification. Notably,
both the thermal arc and silent discharge predated the `official' discover of
plasma by Sir William Crookes in $1872$.


\subsection{Repetitively-Pulsed Nanosecond Discharges}

\todo[inline]{Should the historic stuff be moved to the lit review?}

For a substantial period of time, these two \todo{is this true?} discharges
represented the range of atmospheric-pressure plasmas (\acs{app}). The thermal
arc, though useful, could not be used on delicate substrates. It had the
additional problem of having relatively little control over its chemical
kinetics. Meanwhile, the \acs{dbd} was relegated to ozone production and polymer
processing (relatively low-value applications). Though the \acs{dbd} had
attractive thermal properties, little else was known about how it operated, and
how to control its properties. As recently as $2007$, the National Academies
noted that ``the full promise of \acs{app}s will be known only if they can be
understood and managed based on fundamental scientific principles at two
extremes--the nanoscopic kinetic level, where selective chemistry occurs, and
the global stability level, likened to aerodynamics.''

Coincident to this work were studies of short, pulsed discharges. Initial work
by J.J. Thompson and ??? \todo{who was original?} was driven by an interest in
how breakdown occurred. As shorter pulsers were developed, a number of
researchers became interested in the characteristics of pulsed discharges
themselves. Loeb \todo{who else?} noted that the discharge properties were akin
to those observed in lightning leaders. 


\subsection{Diagnostic Difficulties}

Discuss the information that is lacking. No need to be specific, but be clear
about challenges

\subsection{Research Plan}

Propose research to fill this gap
Specific and cited history of PNDS and related measurements.

\section{Basic Theory}

Basic theory of gaseous breakdown.
