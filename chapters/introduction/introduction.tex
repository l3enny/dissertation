\section{Background}
It is difficult to definitely define the discovery of plasma. The observation of
plasmas has spanned the whole of human existence, with the very moment we began
to gaze at the stars. Wikipedia credits Sir William Crookes with first
identifying it as a new state of matter in $1872$ (though Crookes credits
Faraday with hypothesizing its existence as early as $1819$). Curiously, the
first applications of plasma preceded its discovery as a new state of matter by
$15$ years when Ernst Werner von Siemens described the use of a ``silent
discharge'' in air for the preparation of ozone in the treatment of biologically
contaminated water.

While the study of plasmas possesses a great deal of significance for
astrophysics, it has always been undeniably linked to human applications.  To
this day, studies continue to explore the use of plasma to treat water, now with
concern for industrial pollutants. In general, the last decade has seen a surge
of research on the application of atmospheric-pressure plasmas. However, the
formation of this subset of plasmas has always been shrouded in a degree of
mystery.

The theory for traditional diagnostics is often ill-suited for the increase in
neutral collision, while the instruments themselves are often too slow to
capture the necessary physics. Separately, there has been a lack in motivation.
For many years, the dominant application in plasma research has been fusion
power at the neglect of low temperature plasmas. However, as the possibility for
new and exciting opportunities has become apparent, it has become necessary to
expand on our understanding of atmospheric-pressure plasmas.

My work is concerned with the study of one such plasma. In the literature, it
has been given a variety of unimaginative names that do not reflect its unique
characteristics. As I am unlikely to change this trend, I will adopt the name
`repetitively-pulsed nanosecond discharge' (\ac{rpnd})

My work concerns the study of one such atmospheric-pressure plasma. In the
literature, it goes by several names; here, I will simply refer to it as a
repetitively-pulsed nanosecond discharge (\textsc{rpnd}). Of course, the choice
of a name is only moderately useful, more important is a description. The
\textsc{rpnd} is a plasma that exhibits exceptional uniformity and volume, with
minimal gas heating.

\section{Theory}

\section{Literature Review}

