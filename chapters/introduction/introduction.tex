\section{Background}

\subsection{History of Atmospheric-Pressure Discharges}

Like most physical phenomena, plasmas are typically only described under ideal
circumstances. This means that neutral collisions, and subsequently, atmospheric
plasmas, are often ignored. Neutral collisions tend to obscure the
electromagnetic effects that distinguish a plasma from a gas. However, the
history of observation and study of plasmas is indelibly linked to atmospheric
plasmas. Lightning and static sparks are the most prevalent plasmas on earth.
Indeed, the first artificial plasma was an atmospheric arc, the work of a
Russian scientist named Vasilii Petrov.

The work of Petrov was the forerunner to the study of thermal plasmas. In 1802,
Volta's recent invention of the voltaic pile provided the first source of
constant electrical energy. Using a series of voltaic cells, Petrov was able to
draw the first electrical arc between two sticks of carbon. Aside from its
blinding light, these arcs were characterized by their significant ionization,
and high degree of thermal equilibrium. Gas temperatures could reach thousands
of kelvin.

In contrast, later work by Werner von Siemens, led to the discovery of the
so-called ``silent discharge.'' In recent years, the terminology has changed and
this type of discharge is now referred to as a dielectric-barrier discharge, or
\acs{dbd}. The \acs{dbd} was significantly different from the thermal arc.
Visually, it was much dimmer, and appeared to be composed of many thousands of
individual filaments. Additionally, the \acs{dbd} did not significantly heat the
air, unlike the thermal arc. Finally, the \acs{dbd} was used in the first
commercial plasma application: ozone generation and water purification. Notably,
both the thermal arc and silent discharge predated the `official' discover of
plasma by Sir William Crookes in $1872$.


\subsection{Repetitively-Pulsed Nanosecond Discharges}

For a substantial period of time, these two discharges represented the range of
atmospheric-pressure plasmas (\acs{app}). The thermal arc, though useful, could
not be used on delicate substrates. It had the additional problem of having
relatively little control over its chemical kinetics. Meanwhile, the \acs{dbd}
was relegated to ozone production and polymer processing (relatively low-value
applications). Though the \acs{dbd} had attractive thermal properties, little
else was known about how it operated, and how to control its properties. As
recently as $2007$, the National Academies noted that ``the full promise of
\acs{app}s will be known only if they can be understood and managed based on
fundamental scientific principles at two extremes--the nanoscopic kinetic level,
where selective chemistry occurs, and the global stability level, likened to
aerodynamics.''




\subsection{Diagnostic Difficulties}

Discuss the information that is lacking. No need to be specific, but be clear
about challenges

\subsection{Research Plan}

Propose research to fill this gap

\section{Literature Review}

Specific and cited history of PNDS and related measurements.

\section{Basic Theory}

Basic theory of gaseous breakdown.
