As early as 2001, researchers have proposed the use of a novel, hybrid engine
design for use in supersonic and hypersonic flight \cite{Macheret2001}. In some
ways similar to an earlier program \cite{Gurijanov1996}, it suggested that
magnetohydrodynamic (\acs{mhd}) accelerators were an enabling technology for
hypersonic transport. Briefly, a \acs{mhd} accelerator could be used to
simultaneously produce energy and slow the inlet airflow. This would allow the
use of a conventional turbojet engine at speeds well above its normal
operational range.

However, \acs{mhd} accelerators require an ionized fluid flow. Even at the high
altitudes associated with hypersonic flight, this is not a trivial requirement.
Originally, Macheret suggested the use of electron beams, carefully tuned to
coincide with the peak in the ionization cross section. However, the use of
electron beams in the ionization of high pressure gases is accompanied by a
large number of technical issues, similar to those of some excimer lasers.
Therefore, in 2002, Macheret et al.\ proposed the use of a \acs{rpnd} to produce
an ``electron beam'' in situ \cite{Macheret2002} akin to the beams observed in
certain \acs{fiw} studies. 

Previous studies of \acs{fiw}s in air have observed fast gas heating of
molecular systems \cite{Popov2011}. Up to 40\% of the input energy can be
converted into translational energy through dissociation of oxygen and quenching
or electronically excited nitrogen states. As the \acs{rpnd} physics are very
similar to that of the \acs{fiw}, there is the possibility that it may also
cause fast gas heating. In combustion, this can play an important role in the
chemistry, flame holding, and ignition delay. More generally, gas heating can
impact material processing and ionization efficiency. As such, it is important
to develop reliable temperature diagnostics for \acs{rpnd}s in molecular gases.

This appendix records the development of such a diagnostic for a moderate
pressure \acs{rpnd} in air. The approach used measured the rotational spectra of
a molecular system and used this information to infer the rotational
temperature. As a result of the close energy-spacing of the rotational levels,
this temperature is usually a good measure of the translational temperature of
the system \cite{Laux1993}. This technique is limited by the ability to detect
light from the transitions, and by the equilibration time of the translational
and rotational temperatures.

The measurement of rotational transitions is a common diagnostic for the
measurement of gas temperatures, particularly in the field of combustion.
Matching of the rotational spectra is typically accomplished with a computer
program such as Specair \cite{Laux2002} and LIFBASE \cite{Luque1999}. However, a
survey of the available programs revealed little documentation about the
calculation methods and none which provided the necessary flexibility how the
spectra were generated. This necessitated the development of a program to
automate the generation and matching of rotational spectra.

\section{Experiment}



\section{Theory}



\section{Results}




