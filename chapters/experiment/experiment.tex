\section{Discharge Apparatus}
The discharge apparatus geometry was consistent with the design of a
coaxial transmission line. This is similar to the design guidelines
provided by Vasilyak \cite{Vasilyak1994}. The inner conductor is the
plasma generated by the fast ionization wave. Surrounding that is a
coaxial dielectric, in this case a quartz tube with 2.75" Conflat
flanges on either side. Finally, surrounding the dielectric is the outer
conductor or shield. In this case, the shield was an aluminum cylinder
with slits of approximately 1.5" by 12" milled lengthwise. Figure
\ref{fig:appphoto} \missingfigure{} is a photograph of this discharge
apparatus.

One flange of the quartz tube was held at ground potential, while the
other flange was pulsed to approximately 7 kV. Given that the plasma
undergoes significant decay between pulses, it is assumed that the
impedance is almost infinite when the pulse is first applied, thus the
actual voltage on the powered electrode is likely closer to 14 kV. The
aluminum shield provides the ground connection for the ground electrode.
The two were connected using a copper shim and a compressive shaft
collar. The aluminum tube was connected to a second ground shield with a
one inch copper braid. This second shield was made of copper and was
separated by a teflon cylinder, with walls approximately 1" in
thickness, from the powered electrode. Figure \ref{fig:appschem}
\missingfigure{} is a schematic of the discharge apparatus.

Connected to the powered electrode was a Conflat nipple and an angled
quartz window used in the \acs{lcif} experiments. A short,
silicone-coated, high voltage wire connected the window flange to the
central conductor of an HN connector. The HN connector was seated on a
square copper plate, which was pressed against the shield using four
10-32 screws.

The HN connector was used to attach the transmission line from the high
voltage pulser. Initial experiments attempted to use N connectors,
however these were susceptible to breakdown in the air gap which
separated the center conductor from the outer shield. The transmission
line was approximately 15 m in length. Observations, consistent with
calculations, indicated that this provided a window of approximately 140
ns in which to make measurements before the reflected pulse returned to
the system and re-energized the plasma.

Attached grounded flange was a second quartz envelope that isolated the
ground electrode from the pumping section of the apparatus. Connected to
the second quartz envelope was a stainless steel tee, one side of which
was connected to an angled quartz window used for the \acs{lcif}
experiments. The other side of the tee was isolated with an alumina
break from a series of Conflat fittings connected to a roughing pump.
The roughing pump was connected with a shutoff valve, as well as two
bypass lines with inline needle valves for flow regulation.

\section{Measurement Conditions}
\acs{las}, emission, and coupling energy measurements were made at three
different operating pressures. The operating pressures were: 0.3, 0.5,
1.0, 2.0, 3.0, 4.0, 8.0, and 16.0 Torr. Pressures below 10.0 Torr were
measured with a capacitance manometer with a full scale range of 10.0
Torr, above this a capacitance manometer with a full scale range of
100.0 Torr was used.

Optical measurements were made at three locations along the axis of the
discharge. The measurement location closest to the anode was separated
from it by a distance of approximately six inches. Each other optical
measurement location was moved further from the anode by an additional
three inches.

For each operating condition, measurements were made of the voltage and
current. The voltage measurement was made via an internal divider from
the power supply. Current measurements were made using an back-current
shunt located at a break in the outer shield of the transmission line.
The back-current shunt can be seen in Figure \ref{fig:bcs}. It is
composed of nine, low impedance, one ohm resistors, connected in
parallel. Each side of the resistors were soldered to a piece of copper
foil which was then soldered to the outer shield. A calibrated DC power
supply was used to measure the resistance of the current shunt.

All measurements were made using a LeCroy Waverider oscilloscope with a
bandwidth of 1 GHz. Connections were made using minimal lengths of RG
50/U cable. When necessary for timing purposes, the cable lengths were
matched.  Connections were made using minimal lengths of RG 50/U cable.
When necessary for timing purposes, the cable lengths were matched. All
measurements which required maximum bandwidth were made with a using
external 50 ohm terminators.

\section{Energy Coupling}
For comparison to other discharges, estimates of the energy coupling
were made using the current and voltage characteristics at each
operating pressure.

\section{Absorption Setup}
The \acs{las} setup was based upon the used of a distributed-feedback
laser diode. Temperature and current control of the diode provided
coarse and fine tuning, respectively, for the output frequency. It was
found that it was unnecessary to adjust the temperature for the diode
once the correct transition was found, therefore all tuning was
accomplished using current tuning.

The laser diode was produced by Toptica Photonics (model
#LD-1083-0070-DFB-1), and had a nominal operating power of 70 mW at a
center wavelength of 1083 nm. The diode was held inside a Toptica DL-100
diode housing which contained an integral thermoelectric cooler and
collimating optics. The operation of the diode was controlled by a
Toptica DC 110 monitor, DCC 110 current control, DTC 110 temperature
control, and SC 110 scan control.

A schematic of the optical layout for the absorption experiment can be
seen in Figure \ref{fig:abslayout}. Immediately after exiting the
housing, the beam was passed through an optical isolator in order to
prevent instabilities from back reflections. Next the beam was
attenuated using a neutral density filter in order to keep its intensity
below the saturation level for the transition. Following that, the beam
passed through two apertures for alignment. Here, the beam was split by
a partially reflecting mirror. Approximately 98\% of the beam was
allowed to pass through to a reference photodiode (Thorlabs DET300).
After passing through the plasma, entered another aperture to limit
near-coincident plasma emissions. The background emissions were further
reduced using a long pass filter with a cutoff of 1000 nm. Finally, the
beam was coupled into an optical fiber which connected to the detection
electronics.

The transmitted laser light was detected with an InGaAs photodiode
(Thorlabs DET410). The signal from the diode was often too small to
detect, so the output of the signal photodiode was sent through a
voltage amplifier (Femto HVA-200M-40-B). The light response of this
system is limited by the photodiode which has a nominal rise time of
five nanoseconds. The signal from the amplifier was terminated by a 50
ohm terminator and sensed by the aforementioned oscilloscope.

\subsection{Acquisition Process} The actual acquisition process required
a specific series of steps in order to properly account for all noise
sources. In order to accommodate this process, a custom LabView script
was used to automate the acquisition of the laser transmission spectra.
Generally speaking, the signal can be described as
\begin{equation}
  V_\mathrm{total} = V_\mathrm{signal} + V_\mathrm{background} +
                     V_\mathrm{plasma}.
\end{equation}
In order to remove the background signal, the acquisition scr

\section{Emissions Setup}
