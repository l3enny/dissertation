As was noted in chapter~\ref{chp:introduction}, measurements of the \acs{rpnd}
been mostly limited to the afterglow plasma or time-integrated quantites.
Electric field measurements, either with capacitive probes or nonlinear
wave-mixing, thus far provide the only detail of the \acs{rpnd} during its
development. Though the electric field can be used to estimate electron
densities and reaction rates in the plasma, this requires a number of additional
assumptions.

As a result, there is a lack of reliable information on the particle properties
of the \acs{rpnd} during its development. That said, such information is
essential to confirming the present understanding of how these discharges
develop, optimizing them for specific applications, and providing important
benchmarks for numerical simulations. Therefore, a clear need exists for direct
measurements of the \acs{rpnd} particle properties.

Unfortunately, these measurements present a significant challenge for most
traditional plasma diagnostics. In most situations, the obvious choice would be
the Langmuir probe for its simplicity and ease of implementation. However, the
fast variations in the plasma potential, slow response of the ion, and high
collisionality all preclude this approach \cite{Lieberman2005}. Furthermore, any
physical probe would act as a significant perturbation in the system.

The logical alternative to physical probes is the use of optical diagnostics,
however these have their own associated difficulties. Electrons cannot be
studied by their emissions because, with the exception of bremsstrahlung, they
do not emit light. This leaves the light emitted from excited atoms. Atomic
emission spectroscopy can be used to measure many different plasma quantities,
from electron density, to local electric field strength. Unfortunately,
spontaneous emission can be a slow process compared to the development of the
\acs{rpnd}. For example, the fastest neutral helium transition in visible
wavelengths (3$^3$D$_3$-2$^3$P$_2^\cdot$) has a radiative life of 14 ns
\cite{Ralchenko2011}.

This suggests that instead of waiting for spontaneous emission to occur, it may
be better to use some form of active spectroscopy. Though the added complexity
of a well-characterized light source is undesirable, it adds several interesting
possibilities. For example, Thomson scattering provides a means for direct
interaction with electrons. In addition, it has a high spatial and temporal
resolution and is able to measure the electron density and temperature
simultaneously \cite{VanGessel2012}. However, the electron density limit of
$5\times10^{12}$ cm$^{-3}$ is too low for use with the \acs{rpnd} which may have
electron densities well below this \cite{Pai2009}.

With the ability to directly interact with electrons, the next alternative is to
target one of the excited states of helium. The lowest such state, the triplet
metastable (2$^3$S), resides at 19.82 eV. This is a relatively large energy gap
for an atom and indicates that virtually no such states should exist at room
temperature. The triplet metastable (and all higher-energy states in helium)
will be populated exclusively by energetic electrons. Therefore, a measurement
of the triplet metastable level is a useful indicator of the degree of helium
excitation as the \acs{rpnd} develops, and could potentially be used to infer
some properties of the electron population.

\subsection{Setup}



\subsection{Absorption Analysis}


\subsection{Results}


