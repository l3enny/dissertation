\section{Plasmas}

A volume containing some number of charged particles can be considered a
plasma if it meets three conditions. The first requires that the motion
of charged particles is primarily determined by the electric and
magnetic fields of the volume rather than through collisions with
neutral particles. This is classically expressed by the inequality
\begin{equation}
  \sqrt{n_\mathrm{e} e^2 / (\epsilon_0 m_\mathrm{e})} < \nu,
\end{equation}
where $n_\mathrm{e}$ is the electron density, $e$ is the fundamental
charge, $\epsilon_0$ is the permittivity of free space, $m_\mathrm{e}$
is the mass of an electron, and $\nu$ is the electron-neutral collision
frequency. The left-hand side term is called the electron plasma
frequency, it the characteristic frequency at which a plasma oscillates
in response to a perturbation.

For a sufficiently large number of particles, the behavior of the each
species of the plasma can be described by a continuous probability
distribution function. This function, $f_\alpha(\vec{r}, \vec{v}, t)$,
describes the probability of finding a particle of species $\alpha$, at
position $\vec{r}$, The distribution function for a particle can be
determined by the Vlasov-Fokker-Planck \acs{vfp} equation,
\begin{equation}\label{eq:vfp}
  \frac{\partial f_\alpha}{\partial t} + \vec{v_\alpha}\cdot\nabla f_\alpha +
  q_\alpha \left(\vec{E} + \vec{v_\alpha}\times\vec{B}\right)
  \cdot \nabla_\mathrm{v} f_\alpha = \left( \frac{\partial f_\alpha}
  {\partial t}\right)_\mathrm{coll}.
\end{equation}
Here, $\vec{E}$ is the electric field, $\vec{B}$ is the magnetic field,
and $\partial f_\alpha/(\partial t)_\mathrm{coll}$ is a term
representing all collisions. The \acs{vfp} equation is coupled to
Maxwell's equations in order to obtain a self-consistent description of
the particle distribution and the resulting fields. In essence, this is
the Boltzmann equation from statistical mechanics, however it now
includes several changes. Vlasov replaced the original force term with
the Lorentz equation, and Fokker and Planck introduced the collision
operator on the right-hand side. This is coupled with Maxwell's
equations for a solution of the electric and magnetic fields in the
plasma.

In the absence of external fields and with only elastic collisions, the
equation admits the famous Maxwell-Boltzmann equilibrium distribution,
\begin{equation}\label{eq:mb}
  f_\alpha(v) = n\left(\frac{m_\alpha}{2\pi k_\mathrm{B}T}\right)^{3/2}
    \exp\left(-\frac{m_\alpha v_\alpha^2}{2k_\mathrm{B}T}\right),
\end{equation}
where $n$ is the number of degrees of freedom, $k_\mathrm{B}$ is
Boltzmann's constant, and $T$ is the temperature. A species of particles
which possesses a Boltzmann distribution is said to be in equilibrium.
Likewise, two species with the same distribution are in equilibrium.

Aside from this, the \acs{vfp} equation is notoriously difficult to
solve. As a result, most plasma models use various moments of equation
\ref{eq:vfp} where the velocity dependence has been integrated out.
These moments are the basis for the two-fluid equations, the MHD
formulation, and global models. We will show the first three moments
following the notation of Lieberman and Lichtenberg
\cite{Lieberman2005}. For example, the first moment is the continuity
equation,
\begin{equation}\label{eq:cont}
  \frac{\partial n_\alpha}{\partial t} + \nabla \cdot (n_\alpha \vec{u_\alpha})
  = G_\alpha - L_\alpha,
\end{equation}
where $\vec{u_\alpha}$ is the mean velocity of species $\alpha$,
$G_\alpha$ is its rate of gain, and $L_\alpha$ is the rate of loss. This
equation can be interpreted as the rate of change in particle density
for a particular volume of space.

Though the continuity equation is much simpler than the original
\acs{vfp} equation, it cannot be solved alone. The mean velocity,
$\vec{u}$, is undefined. Typically, this leads to the second moment,
\begin{equation}\label{eq:mom}
  mn_\alpha\left[\frac{\partial \vec{u_\alpha}}{\partial t}
  (\vec{u_\alpha}\cdot \nabla)\right] = q_\alpha n_\alpha(\vec{E} +
  \vec{u_\alpha} \times \vec{B}) - \nabla \cdot \vec{\Pi_\alpha} +
  \vec{f}_{\alpha,\mathrm{coll}}
\end{equation}
where $\vec{\Pi_\alpha}$ is the pressure tensor, and
$\vec{f}_{\alpha,c}$ is the rate of momentum transfer into species
$\alpha$. Again, any solution is stymied by the presence of a new a
term, in this case, $\vec{\Pi_\alpha}$. At this point, an equation of
state can be used to close the set of equations, in this case relating
the pressure to the density. However, later work will benefit from one
more moment.

Following the conservation of momentum, the energy conservation equation
can be derived from the \acs{vfp} equation,
\begin{equation}
  \frac{\partial}{\partial t}\left(\frac{3}{2}p_\alpha\right) 
  + \nabla\cdot\frac{3}{2} (p_\alpha\vec{u}_\alpha)
  + p_\alpha\nabla\cdot\vec{u}_\alpha
  + \nabla\cdot\vec{q}_\alpha
  = \frac{\partial}{\partial
  t}\left(\frac{3}{2}p_\alpha\right)\bigg|_\mathrm{coll}
\end{equation}
where $p_\alpha$ is the species pressure, $q_\alpha$ is the heat flow
vector, and the right-hand side is the time rate of change in energy as
a result of collisions. In our case, we only consider the flux into the
volume (from the electric field) and the distribution of this field via
rate constants. This is the basis for the global model.

\section{Atomic Spectroscopy}

Spectroscopy spans a large body of theory which cannot be adequately
covered here. Given that the measurements are all for helium, we will
limit ourselves to a simple description of atomic spectroscopy. An atom
is made of positively charged nucleus and a number of negatively charged
electrons which orbit this nucleus. In the unperturbed, or ground state,
the electrons occupy orbitals determined by a full solution of the
Schrodinger equation.

However, interactions with other particles or photons can excite one or
more of the electrons into orbitals with higher potential energy. In
most cases relevant to low temperature plasmas, only a single electron
will be excited at any given time. Depending on which orbital the
electron is excited to, it can transition to orbitals with lower
potential energy by emitting a photon. These are typically called
allowed transitions.

Each orbital in an atom can be described by four quantum numbers.
\begin{itemize}
  \item $n$ - The principal quantum number.
  \item $l$ - Orbital angular momentum number.
  \item $j$ - Total angular momentum.
  \item $m_j$ - Total angular magnetic moment.
\end{itemize}
The Pauli exclusion principle restricts more than a single electron from
occupying any given state defined by this series of numbers.
Additionally, each set of numbers determines the potential energy
possessed by an electron in that particular level. 

Allowed transitions are determined by a series of selection rules. These
selection rules can be summed up as the following:
\begin{itemize}
    \item $\Delta S = 0$
    \item $\Delta L = \pm1$
\end{itemize}
Though other transitions are possible (spin and dipole forbidden respectively),
they tend to require an external perturbation in order to induce transition.

Figure shows the what is commonly called a Grotrian diagram for helium.
In this diagram, the vertical axis represents the energy above the
ground state, and the levels are arranged horizontally based on
increasing $L$. Levels which are radiatively linked are connected by
solid lines. As can be seen in this figure, only the levels having $S=1$
are radiatively connected to the ground state. As a result, any helium
atoms that are excited into the triplet manifold tend to stay there,
accumulating in the metastable state, $2^3S$.

Approaching 24 eV, the excited electron enters what is known as the continuum.
The energy separation between states goes as $n^{-2}$, thus at large $n$ the
spacing becomes quite close and the states are almost indistinguishable. The
levels are often referred to as Ryberg states. Above 24.69 eV, the electron
becomes totally detached from the helium nucleus, and all that remains is
a singly ionized helium atom.

Though the emissions of ions can be quite useful in some plasmas, we do not
concern ourselves with them in either the measurements or models. 24.69 eV is
the largest known ionization potential, and as a result, the number of ions and
the emissions associated with them remain relatively small.

\subsection{Spectral Lineshapes}
It is tempting to think that the energy spacing can be calculated exactly,
however there is always some variance about a central energy. This is called the
spectral lineshape, and it effects both the energy of the emitted photon in
radiative transitions, and the photons that an atom can absorb. Though these
variations can be attributed to quantum mechanical effects, the actual result
can derived from the so-called dipole approximation.

In this case, we envision a single electron oscillating about a large, heavy,
positive charge. The full details of this derivation are covered in Siegman
\cite{Siegman1986}, however we'll address some of the most pertinent portions
here. The response of a collection of atoms to an applied electric field can be
expressed as a quantity known as the susceptibility. This is generally defined
as
\begin{equation}
    \tilde{\chi}(\omega) \equiv
    \frac{\tilde{P}(\omega)}{\epsilon_0\tilde{E}(\omega)}
\end{equation}
where $\tilde{\chi}$ is the electric susceptibility, $\tilde{P}$ is the
macroscopic polarization, $\tilde{E}$ is the applied electric field, and
$\omega$ is the frequency of the applied field.

\paragraph{Natural Linewidth}
The electric susceptibility often possesses both a real and imaginary component.
Physically, these respectively represent the reactive and absorptive component
of the medium. Accounting for level-dependent effects, the standard
susceptibility for an atomic transition can be written as
\begin{equation}
    \tilde{\chi}_\mathrm{at}(\omega) = -j\frac{3}{4\pi^2}\frac{\Delta
    N\lambda^3\gamma_\mathrm{rad}}{\Delta\omega_\mathrm{a}}\frac{1}{1
    + 2j(\omega - \omega_\mathrm{a})/\Delta\omega}
\end{equation}
where $\Delta N$ represents the population difference between the upper and
lower levels of the oscillator, $\lambda$ is the transition wavelength,
$\gamma_\mathrm{rad}$ is the natural radiative lifetime of the oscillator,
$\Delta\omega_\mathrm{a}$ is the linewidth of the transition (for an unperturbed
atom, this is simply $\gamma_\mathrm{rad}$), and $\omega_\mathrm{a}$ is the
angular frequency of the transition or oscillator.

This equation is generally known as the complex lorentzian. Separated into its
components it expresses both the absorptive and reactive properties of the
atomic medium. It also clearly susceptible to fields that are displaced from
$\omega_\mathrm{a}$. This is the finite linewidth associated with atomic
emissions and absorption.

This linewidth affects each atom within the medium. Each atom will emit or
absorb radiation with a probability described by this susceptibility.
Consequently, this natural linewidth falls under the homogeneous category of
line broadening.

\paragraph{Pressure Broadening}
Also included in this category is pressure broadening, or more fundamentally,
dephasing. 

\subsection{Radiation Trapping}
