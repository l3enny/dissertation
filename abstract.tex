Though the study of pulsed discharges can be traced back to the ancient Greek
philosophers, there is still much that is not known about them. This work
reports on a new investigation of a repetitively-pulsed nanosecond discharge
(\textsc{rpnd}) in helium over a range of 0.3-16.0 Torr. Several approaches were
used in the study with the emphasis being on spectroscopic techniques and a
global model. Synthesis of the results provided new data and insights on the
development of the \textsc{rpnd}.

Among the results were direct measurements of the triplet metastable states
during the excitation period. This period was found to be unexpectedly long at
low pressures (less than 1.0 Torr), suggesting an excess in high-energy
electrons as compared to an equilibrium distribution. Other phenomena such as a
prominent return stroke and radiation trapping were also identified. Estimates
of the electric field and electron temperatures were obtained for several
conditions. Furthermore, several optical methods of electron temperature
measurement were evaluated for application to the discharge. A ratio of the
3$^3$S-2$^3$P$\odd$ and 3$^1$S-2$^1$P$\odd$ transitions, based on the coronal
model, was found to provide a promising approach.

Overall, these results provide new insight on the development of the
repetitively-pulsed nanosecond discharge. Specifically, they reveal new
information about the excited state dynamics within the discharge, the
non-equilibrium nature of its electrons, and several new approaches for future
studies. This improves the present understanding of repetitively-pulsed
discharges, and advances the knowledge of energy coupling between electric
fields and plasmas.
