This work reports on an investigation of a repetitively-pulsed nanosecond
discharge (\textsc{rpnd}) in helium over a range of 0.3-16.0 Torr. The discharge
was studied experimentally via laser-absorption spectroscopy and optical
emission spectroscopy measurements. In concert with the experimental campaign, a
global model was used to predict the population kinetics and a particle-in-cell
code was used to analyze the \acs{eedf} evolution. Synthesis of the results
provided new data and insights on the development of the \textsc{rpnd}.

Among the results were direct measurements of the triplet metastable states
during the excitation period. This period was found to be unexpectedly long at
low pressures (less than 1.0 Torr), suggesting an excess in high-energy
electrons as compared to an equilibrium distribution. Other phenomena such as a
prominent return stroke and radiation trapping were also identified. Estimates
of the electric field and electron temperatures were obtained for several
conditions. Furthermore, several optical methods for electron temperature
measurement were evaluated for application to the discharge. Based on the global
model simulations, the coronal model was found to apply in the line ratio of the
3$^3$S-2$^3$P$\odd$ and 3$^1$S-2$^1$P$\odd$ transitions, however further work is
needed to ascertain its applicability to experimental discharges.

These results provide new insight on the development of the repetitively-pulsed
nanosecond discharge. Specifically, they reveal new information about the
excited state dynamics within the discharge, the non-equilibrium nature of its
electrons, and several new approaches for future studies. This study extends the
present understanding of repetitively-pulsed discharges, and advances the
knowledge of energy coupling between electric fields and plasmas.
