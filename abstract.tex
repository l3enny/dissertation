The study of pulsed discharges can be traced back to the ancient Greek
philosophers and the sparks they generated with amber and fur. However, their
properties present a number of experimental challenges, and as a result there is
still much that is not known about these types of plasma. This work reports on
an experimental and numerical investigation of a repetitively-pulsed nanosecond
discharge in helium. Among the experimental diagnostics were current and voltage
measurements, laser-absorption spectroscopy of the triplet metastable (2$^3$S)
atomic state, and optical emission spectroscopy. The outcomes of these
measurements were compared to a global model which accounted for 32 separate
species, and 535 individual transitions.

The peak metastable densities were found to range from 1.2-1.8 $\times10^{17}$
m$^{-3}$ with little variation with respect to pressure. The global model was
able to produce excellent matches to the measured metastable densities, provided
the assumption of excitation period lasting longer than the applied pulse. This
extended excitation period is believed to be the result of a return stroke
(verified via optical emissions) and potential contributions from a substantial
population of high-energy electrons. Additional analysis of the optical
emissions revealed the effects of radiation trapping, for all pressures
considered (0.3-16.0 Torr).

Estimates of the peak electric field generated by the global model ranged from
150-350 Td, comparable to capacitive probe measurements from other studies.
However, the accuracy of these results is questionable as the discharge is far
from equilibrium. Likewise, several optical methods of electron temperature
measurement were evaluated for application to the discharge. The most promising
of these was based on a ratio of the emissions from the ratio of the
3$^3$S-2$^3$P$\odd$ transition to the 3$^1$S-2$^1$P$\odd$ transition. Additional
study is required to address the accuracy and applicability of this approach.

Overall, these results provide new insight on the development of the
repetitively-pulsed nanosecond discharge. Specifically, they reveal new
information about the excited state dynamics within the discharge, the
non-equilibrium nature of its electrons, and several new approaches for future
studies. This improves the present understanding of repetitively-pulsed
discharges, and advances the knowledge of energy coupling between electric
fields and plasmas.
