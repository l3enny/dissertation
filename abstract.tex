This work reports on an investigation of a repetitively-pulsed nanosecond
discharge (\textsc{rpnd}) in helium over a range of 0.3-16.0 Torr. The discharge
was studied experimentally via laser-absorption spectroscopy and optical
emission spectroscopy measurements. In concert with the experimental campaign, a
global model of a helium plasma was used to predict the population kinetics and
a particle-in-cell code was used to analyze the \textsc{eedf} evolution.
Synthesis of the results provided new data and insights on the development of
the \textsc{rpnd}.

Among the results were direct measurements of the triplet metastable states
during the excitation period. This period was found to be unexpectedly long at
low pressures (less than 1.0 Torr), suggesting an excess in high-energy
electrons as compared to an equilibrium distribution. Other phenomena such as a
prominent return stroke and a additional energy deposition by reflections in the
transmission line were also identified. Estimates of the electric field and
electron temperatures were obtained for several conditions. Furthermore,
